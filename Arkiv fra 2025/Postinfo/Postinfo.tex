\title{Postinfo}\\
Kære \textbf{Nørre Uttrup}\\
\newline
Velkommen på Sagaløbet 2025.\\
Vi har glædet os vildt til at lave løb med jer.\\
\newline
I har fået \textbf{Post A1}. Posten er placeret på \textbf{RUTE A}\\
Vi har, som mange tidligere år, forsøgt at gøre løbet bedre for både deltagere og jer. I år drejer det sig om småjusteringer, så hvis I har været med før, så er det meste, som I kender det.\\
\newline
Hvis I ikke har været med før, så håber vi, at dette infobrev kan hjælpe jer med at være klædt godt på. Har I spørgsmål efter I har læst brevet, så sig til! Ring til mål på tlf. +45 3267 9499.
\subsection{Sagaløbets principper}
Til planlægningen af årets løb har vi skabt seks principper, som er det bærende element for både planlægning, afviklingen og udviklingen af Sagaløbet.\\
De seks principper er:
\begin{itemize}
  \item \textbf{Vi skaber Sagaløbet sammen}
  \item \textbf{Alle skal opleve at lykkes}
  \item \textbf{Alle skal opleve udfordringer}
  \item \textbf{Det kendte og ukendte}
  \item \textbf{Vi gør os umage}
  \item \textbf{Retfærdig konkurrence}
\end{itemize}
Vi håber, at I kan genkende at Sagaløbet er bygget op om disse seks bærende principper. Det bliver også med disse seks principper, at vi kommer til at træffe beslutninger under løbet. Har I spørgsmål til principperne, er vi altid klar til at tage en snak om dem.
\subsection{Ruter og deltagerstrøm}
Løbet er inddelt i fire mindre ruter. Hver rute er angivet med et bogstav og har 4 levende poster og én død post. Derudover er der 4 levende seniorposter i målområdet.\\
Ved at inddele løbet i fire mindre løb kan vi sprede deltagerne ud, have kontakt med dem løbende, give dem et helle og sikre et bedre flow i løbet.\\
\newline
Det betyder at I skal stå post fra 16:00 til 08:00.\\
Det er lang tid, og det kan være nødvendigt at hjælpe hinanden med at holde pauser undervejs.\\
\newline
Holdene er delt op i fire grupper med 12-13 hold i hver gruppe. Det betyder at I forhåbentlig når at se alle holdene fra en gruppe, før den næste dukker op.\\
For at holdet har ret til at løse jeres post, skal de være igang med jeres rute, og skal have løst de foregående ruter. I kan se om ruterne før jeres egen, har fået et hul på holdenes ID-kort.
\subsection{App og tjek ind}
På www.post25.sagalobet.dk skal I indberette holdenes point samt tjekke deltagerne ind og ud.\\
I skal logge ind med jeres postnummer - Altså \textbf{A1}\\
\newline
Kommer I til at lave en forkert registrering, skal I kontakte mål med det samme, på tlf. 3267 9499.
\newpage
\vspace*{.4cm}
\subsection{Postbeskrivelser}
I har fået udleveret 5 plader med postbeskrivelser til jeres post. På hver side af pladen er der en postbeskrivelse til enten væbner- eller seniorhold. I kan se på bjælken i toppen, om det er til væbner- eller seniorhold. Vi har i udgangspunktet ikke lavet andet end sproglige tilretninger på posten, men læs den lige igennem, så I er sikre på at vide det samme som deltagerne ved.\\
Vi vil \textbf{meget gerne} have pladerne retur i målområdet efter løbet.
\subsection{Sikkerhed på løbet}
For at give deltagerne en tryg oplevelse har vi nogle sikkerhedskrav til jeres post:
\begin{itemize}
  \item I skal opsætte jeres velkomstskilt, så det er det første deltagerne ser, når de kommer tæt på posten.
  \item I skal alle sammen bære de udleverede ID-kort yderst, så de er tydelige for deltagerne.
  \item I skal udlevere postbeskrivelserne til deltagerne, så de ved, at I er en ægte post.
  \item I skal huske at tjekke deltagerne ind og ud så snart de ankommer og forlader posten, så vi kan følge med i, hvor langt de er.
\end{itemize}
\subsection{Point}
Alle levende poster giver mellem 0 og 100 point, med mulighed for at optjene 25 bonuspoint, hvis posten løses på en bestemt måde. Giv kun deltagerne point ud fra deres opgaveløsning og ikke ud fra om de fedter for jer. Når et hold har gennemført posten, skal pointene indberettes i appen. Husk, at I ikke kan lave om i jeres registreringer, og det kun er muligt at indsende point én gang pr. hold.\\
\newline
Kommer I til at lave en forkert registrering, skal I kontakte mål med det samme, på tlf. 3267 9499.
\subsection{Åbningstider}
Alle poster åbner kl. 16:00 og lukker kl. 08:00.\\
Tjekker et hold ind på jeres post inden kl. 08.00, må de stadig gerne løse posten.\\
Hold, som ankommer efter kl. 08:00, kan ikke løse posten.\\
Basen lukker også kl. 08:00.
\subsection{Postforløb}
For at give deltagerne den samme oplevelse, vil vi gerne have jer til at følge dette postforløb:
\begin{itemize}
  \item Byd holdet velkommen til posten.
  \item Modtag holdets ID-kort og tjek dem ind på app\'en.
  \item Udlever postbeskrivelsen til holdet, hvis der ikke er kø, eller de er forrest i køen. Hvis holdet skal sidde i kø, skal I fortælle dem, hvor lang tid de kan forvente at vente.
  \item Sæt holdet i gang med at løse posten, når de har læst postbeskrivelsen.
  \item Fortæl holdet, hvor mange point de har tjent, når de er færdige med at løse posten.
  \item Registrer holdets point på app\'en.
  \item Tjek holdet ud på app\'en og giv dem deres ID-kort tilbage.
  \item Hold øje med om holdet er ved at gå kold. Giv dem noget opbakning, hvis det er tilfældet.
\end{itemize}
\newpage
\vspace*{.4cm}
\subsection{Baseområdet}
Hvis der er brug for at sende noget af jeres mandskab ind på en pause, kan de finde sig til rette i postområdet på basen. Husk at holdene kan komme hele natten, så sørg for ikke at sende hele postens mandskab afsted på en gang.\\
På basen er der en del af området, som er forbeholdt udvalget. Vi vil meget gerne hyggesnakke med jer og svare på spørgsmål, men vi har brug for ro i udvalgsområdet. Det håber, vi at I vil respektere. Har I brug for at nogen fra udvalget kommer ud, kan I sige det til de udvalgsmedlemmer, som bemander basebordet.
\subsection{Hvis en deltager vil udgå af løbet}
Undervejs på løbet kommer I til at opleve mange trætte børn og unge. Nogle af dem vil måske også gerne give op. Hvis det er tilfældet, skal I først spørge om de er sikre. Fortæl deltagerne, hvor langt der ca. er til næste post og hvem der bemander den. Det kan være, at det er deres egen kreds og at de gerne vil nå forbi den post også. Hvis deltageren stadig ønsker at give op, skal I finde et sted, de kan holde sig i læ og så ringe til mål på tlf. 3267 9499. Efterfølgende henter vi deltageren, når det er muligt. Deltagere som ønsker at give op, må gerne hjælpe sit hold med at løse posten.\\
\newline
Oplever I deltagere, hvor I selv tænker, at det vil være bedst for deltageren at udgå at løbet, så kontakt udvalget. Det er kun deltagerne eller udvalget der kan træffe beslutning om at udgå af løbet.
\subsection{Følg med i løbet}
I app\'en kan I se, hvor langt holdene er nået og se om der er nogen på vej til jeres post.
\subsection{Postpokalen}
Alle løbets levende poster bliver vurderet af deltagerne. Posterne får point på en skala fra 1-10.\\
Den samlede score udregnes som et gennemsnit af de givne point. For at vi får den bedst mulige konkurrence, må I meget gerne minde deltagerne om at give jer point til postpokalen, inden de tjekker ud.\\
Undgå også at fedte for deltagerne med slik, snacks eller andet. - I giver heller ikke deltagerne fedtepoint for at fedte for jer.\\
\newline
Udover deltagerne, vil forskellige jury-medlemmer dukke op i løbet af løbet for at vurdere jeres post. Vi håber at I vil tage godt imod dem.
\subsection{Hemmeligheder}
En stor del af de point, deltagerne kan opnå på Sagaløbet, findes som hemmelige og skjulte point. Opdager I sådan nogle, så lad være med at fortælle deltagerne om dem - også de deltagere fra jeres egen kreds. På den måde får vi den mest fair konkurrence.
\newpage
\vspace*{.4cm}
\subsection{Mærker}
Når vi er færdige med løbet, kan alt postmandskab, som ikke allerede har et mærke, få et SagaCrew-mærke.\\
Alle deltagere som gennemfører Sagaløbet får også et mærke. Deltagere som tidligere har gennemført løbet, men kun fået ét mærke, kan købe ekstra mærker efter præmieoverrækkelsen til 10 kr. stykket.
\subsection{Kontakt}
Får I brug for at komme i kontakt med udvalget gennem natten, så ring til mål på tlf. 3267 9499. Har I brug for at få fat på et bestemt udvalgsmedlem, kan vores kontaktoplysninger findes på hjemmesiden. Vi ser dog helst at I kontakter mål frem for at kontakte enkelte udvalgsmedlemmer, da vi ordner koordinationen herfra.
\newline
Til sidst er der ikke så meget andet at sige end tak!\begin{itemize}
  \item fordi I er med til at skabe Sagaløbet 2025
  \item for at være gode ambassadører for vores arrangement
  \item for at tage jeres væbnere, seniorvæbnere og seniorer med på løb
  \item for at sikre at Sagaløbet er en fair konkurrence
  \item for at give deltagerne uforglemmelige oplevelser, som giver dem mulighed for at vokse
  \item for at hjælpe os med at udvikle løbet
  \item for at give os muligheden for at skabe fantastiske FDF-oplevelser i fællesskab
  \item for alt det vi ikke får sagt tak for
\end{itemize}
Vi glæder os helt vildt til at besøge jer i løbet af natten, og til at opleve løbet sammen med jer og deltagerne!
\newline
Rigtig godt løb!\\
\newline
\textcolor{søblå}{De bedste hilsner}\\
\textcolor{natblå}{\textbf{Alberte, Augusta, Charlotte, Dan, Jens Peter, Jonas, Lasse, Lucas, Magnus \& Thor}}\\
\textcolor{natblå}{\textbf{Sagaløbsudvalget}}\\
\newpage
\title{Postinfo}\\
Kære \textbf{Øster Hornum}\\
\newline
Velkommen på Sagaløbet 2025.\\
Vi har glædet os vildt til at lave løb med jer.\\
\newline
I har fået \textbf{Post A2}. Posten er placeret på \textbf{RUTE A}\\
Vi har, som mange tidligere år, forsøgt at gøre løbet bedre for både deltagere og jer. I år drejer det sig om småjusteringer, så hvis I har været med før, så er det meste, som I kender det.\\
\newline
Hvis I ikke har været med før, så håber vi, at dette infobrev kan hjælpe jer med at være klædt godt på. Har I spørgsmål efter I har læst brevet, så sig til! Ring til mål på tlf. +45 3267 9499.
\subsection{Sagaløbets principper}
Til planlægningen af årets løb har vi skabt seks principper, som er det bærende element for både planlægning, afviklingen og udviklingen af Sagaløbet.\\
De seks principper er:
\begin{itemize}
  \item \textbf{Vi skaber Sagaløbet sammen}
  \item \textbf{Alle skal opleve at lykkes}
  \item \textbf{Alle skal opleve udfordringer}
  \item \textbf{Det kendte og ukendte}
  \item \textbf{Vi gør os umage}
  \item \textbf{Retfærdig konkurrence}
\end{itemize}
Vi håber, at I kan genkende at Sagaløbet er bygget op om disse seks bærende principper. Det bliver også med disse seks principper, at vi kommer til at træffe beslutninger under løbet. Har I spørgsmål til principperne, er vi altid klar til at tage en snak om dem.
\subsection{Ruter og deltagerstrøm}
Løbet er inddelt i fire mindre ruter. Hver rute er angivet med et bogstav og har 4 levende poster og én død post. Derudover er der 4 levende seniorposter i målområdet.\\
Ved at inddele løbet i fire mindre løb kan vi sprede deltagerne ud, have kontakt med dem løbende, give dem et helle og sikre et bedre flow i løbet.\\
\newline
Det betyder at I skal stå post fra 16:00 til 08:00.\\
Det er lang tid, og det kan være nødvendigt at hjælpe hinanden med at holde pauser undervejs.\\
\newline
Holdene er delt op i fire grupper med 12-13 hold i hver gruppe. Det betyder at I forhåbentlig når at se alle holdene fra en gruppe, før den næste dukker op.\\
For at holdet har ret til at løse jeres post, skal de være igang med jeres rute, og skal have løst de foregående ruter. I kan se om ruterne før jeres egen, har fået et hul på holdenes ID-kort.
\subsection{App og tjek ind}
På www.post25.sagalobet.dk skal I indberette holdenes point samt tjekke deltagerne ind og ud.\\
I skal logge ind med jeres postnummer - Altså \textbf{A2}\\
\newline
Kommer I til at lave en forkert registrering, skal I kontakte mål med det samme, på tlf. 3267 9499.
\newpage
\vspace*{.4cm}
\subsection{Postbeskrivelser}
I har fået udleveret 5 plader med postbeskrivelser til jeres post. På hver side af pladen er der en postbeskrivelse til enten væbner- eller seniorhold. I kan se på bjælken i toppen, om det er til væbner- eller seniorhold. Vi har i udgangspunktet ikke lavet andet end sproglige tilretninger på posten, men læs den lige igennem, så I er sikre på at vide det samme som deltagerne ved.\\
Vi vil \textbf{meget gerne} have pladerne retur i målområdet efter løbet.
\subsection{Sikkerhed på løbet}
For at give deltagerne en tryg oplevelse har vi nogle sikkerhedskrav til jeres post:
\begin{itemize}
  \item I skal opsætte jeres velkomstskilt, så det er det første deltagerne ser, når de kommer tæt på posten.
  \item I skal alle sammen bære de udleverede ID-kort yderst, så de er tydelige for deltagerne.
  \item I skal udlevere postbeskrivelserne til deltagerne, så de ved, at I er en ægte post.
  \item I skal huske at tjekke deltagerne ind og ud så snart de ankommer og forlader posten, så vi kan følge med i, hvor langt de er.
\end{itemize}
\subsection{Point}
Alle levende poster giver mellem 0 og 100 point, med mulighed for at optjene 25 bonuspoint, hvis posten løses på en bestemt måde. Giv kun deltagerne point ud fra deres opgaveløsning og ikke ud fra om de fedter for jer. Når et hold har gennemført posten, skal pointene indberettes i appen. Husk, at I ikke kan lave om i jeres registreringer, og det kun er muligt at indsende point én gang pr. hold.\\
\newline
Kommer I til at lave en forkert registrering, skal I kontakte mål med det samme, på tlf. 3267 9499.
\subsection{Åbningstider}
Alle poster åbner kl. 16:00 og lukker kl. 08:00.\\
Tjekker et hold ind på jeres post inden kl. 08.00, må de stadig gerne løse posten.\\
Hold, som ankommer efter kl. 08:00, kan ikke løse posten.\\
Basen lukker også kl. 08:00.
\subsection{Postforløb}
For at give deltagerne den samme oplevelse, vil vi gerne have jer til at følge dette postforløb:
\begin{itemize}
  \item Byd holdet velkommen til posten.
  \item Modtag holdets ID-kort og tjek dem ind på app\'en.
  \item Udlever postbeskrivelsen til holdet, hvis der ikke er kø, eller de er forrest i køen. Hvis holdet skal sidde i kø, skal I fortælle dem, hvor lang tid de kan forvente at vente.
  \item Sæt holdet i gang med at løse posten, når de har læst postbeskrivelsen.
  \item Fortæl holdet, hvor mange point de har tjent, når de er færdige med at løse posten.
  \item Registrer holdets point på app\'en.
  \item Tjek holdet ud på app\'en og giv dem deres ID-kort tilbage.
  \item Hold øje med om holdet er ved at gå kold. Giv dem noget opbakning, hvis det er tilfældet.
\end{itemize}
\newpage
\vspace*{.4cm}
\subsection{Baseområdet}
Hvis der er brug for at sende noget af jeres mandskab ind på en pause, kan de finde sig til rette i postområdet på basen. Husk at holdene kan komme hele natten, så sørg for ikke at sende hele postens mandskab afsted på en gang.\\
På basen er der en del af området, som er forbeholdt udvalget. Vi vil meget gerne hyggesnakke med jer og svare på spørgsmål, men vi har brug for ro i udvalgsområdet. Det håber, vi at I vil respektere. Har I brug for at nogen fra udvalget kommer ud, kan I sige det til de udvalgsmedlemmer, som bemander basebordet.
\subsection{Hvis en deltager vil udgå af løbet}
Undervejs på løbet kommer I til at opleve mange trætte børn og unge. Nogle af dem vil måske også gerne give op. Hvis det er tilfældet, skal I først spørge om de er sikre. Fortæl deltagerne, hvor langt der ca. er til næste post og hvem der bemander den. Det kan være, at det er deres egen kreds og at de gerne vil nå forbi den post også. Hvis deltageren stadig ønsker at give op, skal I finde et sted, de kan holde sig i læ og så ringe til mål på tlf. 3267 9499. Efterfølgende henter vi deltageren, når det er muligt. Deltagere som ønsker at give op, må gerne hjælpe sit hold med at løse posten.\\
\newline
Oplever I deltagere, hvor I selv tænker, at det vil være bedst for deltageren at udgå at løbet, så kontakt udvalget. Det er kun deltagerne eller udvalget der kan træffe beslutning om at udgå af løbet.
\subsection{Følg med i løbet}
I app\'en kan I se, hvor langt holdene er nået og se om der er nogen på vej til jeres post.
\subsection{Postpokalen}
Alle løbets levende poster bliver vurderet af deltagerne. Posterne får point på en skala fra 1-10.\\
Den samlede score udregnes som et gennemsnit af de givne point. For at vi får den bedst mulige konkurrence, må I meget gerne minde deltagerne om at give jer point til postpokalen, inden de tjekker ud.\\
Undgå også at fedte for deltagerne med slik, snacks eller andet. - I giver heller ikke deltagerne fedtepoint for at fedte for jer.\\
\newline
Udover deltagerne, vil forskellige jury-medlemmer dukke op i løbet af løbet for at vurdere jeres post. Vi håber at I vil tage godt imod dem.
\subsection{Hemmeligheder}
En stor del af de point, deltagerne kan opnå på Sagaløbet, findes som hemmelige og skjulte point. Opdager I sådan nogle, så lad være med at fortælle deltagerne om dem - også de deltagere fra jeres egen kreds. På den måde får vi den mest fair konkurrence.
\newpage
\vspace*{.4cm}
\subsection{Mærker}
Når vi er færdige med løbet, kan alt postmandskab, som ikke allerede har et mærke, få et SagaCrew-mærke.\\
Alle deltagere som gennemfører Sagaløbet får også et mærke. Deltagere som tidligere har gennemført løbet, men kun fået ét mærke, kan købe ekstra mærker efter præmieoverrækkelsen til 10 kr. stykket.
\subsection{Kontakt}
Får I brug for at komme i kontakt med udvalget gennem natten, så ring til mål på tlf. 3267 9499. Har I brug for at få fat på et bestemt udvalgsmedlem, kan vores kontaktoplysninger findes på hjemmesiden. Vi ser dog helst at I kontakter mål frem for at kontakte enkelte udvalgsmedlemmer, da vi ordner koordinationen herfra.
\newline
Til sidst er der ikke så meget andet at sige end tak!\begin{itemize}
  \item fordi I er med til at skabe Sagaløbet 2025
  \item for at være gode ambassadører for vores arrangement
  \item for at tage jeres væbnere, seniorvæbnere og seniorer med på løb
  \item for at sikre at Sagaløbet er en fair konkurrence
  \item for at give deltagerne uforglemmelige oplevelser, som giver dem mulighed for at vokse
  \item for at hjælpe os med at udvikle løbet
  \item for at give os muligheden for at skabe fantastiske FDF-oplevelser i fællesskab
  \item for alt det vi ikke får sagt tak for
\end{itemize}
Vi glæder os helt vildt til at besøge jer i løbet af natten, og til at opleve løbet sammen med jer og deltagerne!
\newline
Rigtig godt løb!\\
\newline
\textcolor{søblå}{De bedste hilsner}\\
\textcolor{natblå}{\textbf{Alberte, Augusta, Charlotte, Dan, Jens Peter, Jonas, Lasse, Lucas, Magnus \& Thor}}\\
\textcolor{natblå}{\textbf{Sagaløbsudvalget}}\\
\newpage
\title{Postinfo}\\
Kære \textbf{Udbudspost}\\
\newline
Velkommen på Sagaløbet 2025.\\
Vi har glædet os vildt til at lave løb med jer.\\
\newline
I har fået \textbf{Post A3}. Posten er placeret på \textbf{RUTE A}\\
Vi har, som mange tidligere år, forsøgt at gøre løbet bedre for både deltagere og jer. I år drejer det sig om småjusteringer, så hvis I har været med før, så er det meste, som I kender det.\\
\newline
Hvis I ikke har været med før, så håber vi, at dette infobrev kan hjælpe jer med at være klædt godt på. Har I spørgsmål efter I har læst brevet, så sig til! Ring til mål på tlf. +45 3267 9499.
\subsection{Sagaløbets principper}
Til planlægningen af årets løb har vi skabt seks principper, som er det bærende element for både planlægning, afviklingen og udviklingen af Sagaløbet.\\
De seks principper er:
\begin{itemize}
  \item \textbf{Vi skaber Sagaløbet sammen}
  \item \textbf{Alle skal opleve at lykkes}
  \item \textbf{Alle skal opleve udfordringer}
  \item \textbf{Det kendte og ukendte}
  \item \textbf{Vi gør os umage}
  \item \textbf{Retfærdig konkurrence}
\end{itemize}
Vi håber, at I kan genkende at Sagaløbet er bygget op om disse seks bærende principper. Det bliver også med disse seks principper, at vi kommer til at træffe beslutninger under løbet. Har I spørgsmål til principperne, er vi altid klar til at tage en snak om dem.
\subsection{Ruter og deltagerstrøm}
Løbet er inddelt i fire mindre ruter. Hver rute er angivet med et bogstav og har 4 levende poster og én død post. Derudover er der 4 levende seniorposter i målområdet.\\
Ved at inddele løbet i fire mindre løb kan vi sprede deltagerne ud, have kontakt med dem løbende, give dem et helle og sikre et bedre flow i løbet.\\
\newline
Det betyder at I skal stå post fra 16:00 til 08:00.\\
Det er lang tid, og det kan være nødvendigt at hjælpe hinanden med at holde pauser undervejs.\\
\newline
Holdene er delt op i fire grupper med 12-13 hold i hver gruppe. Det betyder at I forhåbentlig når at se alle holdene fra en gruppe, før den næste dukker op.\\
For at holdet har ret til at løse jeres post, skal de være igang med jeres rute, og skal have løst de foregående ruter. I kan se om ruterne før jeres egen, har fået et hul på holdenes ID-kort.
\subsection{App og tjek ind}
På www.post25.sagalobet.dk skal I indberette holdenes point samt tjekke deltagerne ind og ud.\\
I skal logge ind med jeres postnummer - Altså \textbf{A3}\\
\newline
Kommer I til at lave en forkert registrering, skal I kontakte mål med det samme, på tlf. 3267 9499.
\newpage
\vspace*{.4cm}
\subsection{Postbeskrivelser}
I har fået udleveret 5 plader med postbeskrivelser til jeres post. På hver side af pladen er der en postbeskrivelse til enten væbner- eller seniorhold. I kan se på bjælken i toppen, om det er til væbner- eller seniorhold. Vi har i udgangspunktet ikke lavet andet end sproglige tilretninger på posten, men læs den lige igennem, så I er sikre på at vide det samme som deltagerne ved.\\
Vi vil \textbf{meget gerne} have pladerne retur i målområdet efter løbet.
\subsection{Sikkerhed på løbet}
For at give deltagerne en tryg oplevelse har vi nogle sikkerhedskrav til jeres post:
\begin{itemize}
  \item I skal opsætte jeres velkomstskilt, så det er det første deltagerne ser, når de kommer tæt på posten.
  \item I skal alle sammen bære de udleverede ID-kort yderst, så de er tydelige for deltagerne.
  \item I skal udlevere postbeskrivelserne til deltagerne, så de ved, at I er en ægte post.
  \item I skal huske at tjekke deltagerne ind og ud så snart de ankommer og forlader posten, så vi kan følge med i, hvor langt de er.
\end{itemize}
\subsection{Point}
Alle levende poster giver mellem 0 og 100 point, med mulighed for at optjene 25 bonuspoint, hvis posten løses på en bestemt måde. Giv kun deltagerne point ud fra deres opgaveløsning og ikke ud fra om de fedter for jer. Når et hold har gennemført posten, skal pointene indberettes i appen. Husk, at I ikke kan lave om i jeres registreringer, og det kun er muligt at indsende point én gang pr. hold.\\
\newline
Kommer I til at lave en forkert registrering, skal I kontakte mål med det samme, på tlf. 3267 9499.
\subsection{Åbningstider}
Alle poster åbner kl. 16:00 og lukker kl. 08:00.\\
Tjekker et hold ind på jeres post inden kl. 08.00, må de stadig gerne løse posten.\\
Hold, som ankommer efter kl. 08:00, kan ikke løse posten.\\
Basen lukker også kl. 08:00.
\subsection{Postforløb}
For at give deltagerne den samme oplevelse, vil vi gerne have jer til at følge dette postforløb:
\begin{itemize}
  \item Byd holdet velkommen til posten.
  \item Modtag holdets ID-kort og tjek dem ind på app\'en.
  \item Udlever postbeskrivelsen til holdet, hvis der ikke er kø, eller de er forrest i køen. Hvis holdet skal sidde i kø, skal I fortælle dem, hvor lang tid de kan forvente at vente.
  \item Sæt holdet i gang med at løse posten, når de har læst postbeskrivelsen.
  \item Fortæl holdet, hvor mange point de har tjent, når de er færdige med at løse posten.
  \item Registrer holdets point på app\'en.
  \item Tjek holdet ud på app\'en og giv dem deres ID-kort tilbage.
  \item Hold øje med om holdet er ved at gå kold. Giv dem noget opbakning, hvis det er tilfældet.
\end{itemize}
\newpage
\vspace*{.4cm}
\subsection{Baseområdet}
Hvis der er brug for at sende noget af jeres mandskab ind på en pause, kan de finde sig til rette i postområdet på basen. Husk at holdene kan komme hele natten, så sørg for ikke at sende hele postens mandskab afsted på en gang.\\
På basen er der en del af området, som er forbeholdt udvalget. Vi vil meget gerne hyggesnakke med jer og svare på spørgsmål, men vi har brug for ro i udvalgsområdet. Det håber, vi at I vil respektere. Har I brug for at nogen fra udvalget kommer ud, kan I sige det til de udvalgsmedlemmer, som bemander basebordet.
\subsection{Hvis en deltager vil udgå af løbet}
Undervejs på løbet kommer I til at opleve mange trætte børn og unge. Nogle af dem vil måske også gerne give op. Hvis det er tilfældet, skal I først spørge om de er sikre. Fortæl deltagerne, hvor langt der ca. er til næste post og hvem der bemander den. Det kan være, at det er deres egen kreds og at de gerne vil nå forbi den post også. Hvis deltageren stadig ønsker at give op, skal I finde et sted, de kan holde sig i læ og så ringe til mål på tlf. 3267 9499. Efterfølgende henter vi deltageren, når det er muligt. Deltagere som ønsker at give op, må gerne hjælpe sit hold med at løse posten.\\
\newline
Oplever I deltagere, hvor I selv tænker, at det vil være bedst for deltageren at udgå at løbet, så kontakt udvalget. Det er kun deltagerne eller udvalget der kan træffe beslutning om at udgå af løbet.
\subsection{Følg med i løbet}
I app\'en kan I se, hvor langt holdene er nået og se om der er nogen på vej til jeres post.
\subsection{Postpokalen}
Alle løbets levende poster bliver vurderet af deltagerne. Posterne får point på en skala fra 1-10.\\
Den samlede score udregnes som et gennemsnit af de givne point. For at vi får den bedst mulige konkurrence, må I meget gerne minde deltagerne om at give jer point til postpokalen, inden de tjekker ud.\\
Undgå også at fedte for deltagerne med slik, snacks eller andet. - I giver heller ikke deltagerne fedtepoint for at fedte for jer.\\
\newline
Udover deltagerne, vil forskellige jury-medlemmer dukke op i løbet af løbet for at vurdere jeres post. Vi håber at I vil tage godt imod dem.
\subsection{Hemmeligheder}
En stor del af de point, deltagerne kan opnå på Sagaløbet, findes som hemmelige og skjulte point. Opdager I sådan nogle, så lad være med at fortælle deltagerne om dem - også de deltagere fra jeres egen kreds. På den måde får vi den mest fair konkurrence.
\newpage
\vspace*{.4cm}
\subsection{Mærker}
Når vi er færdige med løbet, kan alt postmandskab, som ikke allerede har et mærke, få et SagaCrew-mærke.\\
Alle deltagere som gennemfører Sagaløbet får også et mærke. Deltagere som tidligere har gennemført løbet, men kun fået ét mærke, kan købe ekstra mærker efter præmieoverrækkelsen til 10 kr. stykket.
\subsection{Kontakt}
Får I brug for at komme i kontakt med udvalget gennem natten, så ring til mål på tlf. 3267 9499. Har I brug for at få fat på et bestemt udvalgsmedlem, kan vores kontaktoplysninger findes på hjemmesiden. Vi ser dog helst at I kontakter mål frem for at kontakte enkelte udvalgsmedlemmer, da vi ordner koordinationen herfra.
\newline
Til sidst er der ikke så meget andet at sige end tak!\begin{itemize}
  \item fordi I er med til at skabe Sagaløbet 2025
  \item for at være gode ambassadører for vores arrangement
  \item for at tage jeres væbnere, seniorvæbnere og seniorer med på løb
  \item for at sikre at Sagaløbet er en fair konkurrence
  \item for at give deltagerne uforglemmelige oplevelser, som giver dem mulighed for at vokse
  \item for at hjælpe os med at udvikle løbet
  \item for at give os muligheden for at skabe fantastiske FDF-oplevelser i fællesskab
  \item for alt det vi ikke får sagt tak for
\end{itemize}
Vi glæder os helt vildt til at besøge jer i løbet af natten, og til at opleve løbet sammen med jer og deltagerne!
\newline
Rigtig godt løb!\\
\newline
\textcolor{søblå}{De bedste hilsner}\\
\textcolor{natblå}{\textbf{Alberte, Augusta, Charlotte, Dan, Jens Peter, Jonas, Lasse, Lucas, Magnus \& Thor}}\\
\textcolor{natblå}{\textbf{Sagaløbsudvalget}}\\
\newpage
\title{Postinfo}\\
Kære \textbf{Viby J}\\
\newline
Velkommen på Sagaløbet 2025.\\
Vi har glædet os vildt til at lave løb med jer.\\
\newline
I har fået \textbf{Post A4}. Posten er placeret på \textbf{RUTE A}\\
Vi har, som mange tidligere år, forsøgt at gøre løbet bedre for både deltagere og jer. I år drejer det sig om småjusteringer, så hvis I har været med før, så er det meste, som I kender det.\\
\newline
Hvis I ikke har været med før, så håber vi, at dette infobrev kan hjælpe jer med at være klædt godt på. Har I spørgsmål efter I har læst brevet, så sig til! Ring til mål på tlf. +45 3267 9499.
\subsection{Sagaløbets principper}
Til planlægningen af årets løb har vi skabt seks principper, som er det bærende element for både planlægning, afviklingen og udviklingen af Sagaløbet.\\
De seks principper er:
\begin{itemize}
  \item \textbf{Vi skaber Sagaløbet sammen}
  \item \textbf{Alle skal opleve at lykkes}
  \item \textbf{Alle skal opleve udfordringer}
  \item \textbf{Det kendte og ukendte}
  \item \textbf{Vi gør os umage}
  \item \textbf{Retfærdig konkurrence}
\end{itemize}
Vi håber, at I kan genkende at Sagaløbet er bygget op om disse seks bærende principper. Det bliver også med disse seks principper, at vi kommer til at træffe beslutninger under løbet. Har I spørgsmål til principperne, er vi altid klar til at tage en snak om dem.
\subsection{Ruter og deltagerstrøm}
Løbet er inddelt i fire mindre ruter. Hver rute er angivet med et bogstav og har 4 levende poster og én død post. Derudover er der 4 levende seniorposter i målområdet.\\
Ved at inddele løbet i fire mindre løb kan vi sprede deltagerne ud, have kontakt med dem løbende, give dem et helle og sikre et bedre flow i løbet.\\
\newline
Det betyder at I skal stå post fra 16:00 til 08:00.\\
Det er lang tid, og det kan være nødvendigt at hjælpe hinanden med at holde pauser undervejs.\\
\newline
Holdene er delt op i fire grupper med 12-13 hold i hver gruppe. Det betyder at I forhåbentlig når at se alle holdene fra en gruppe, før den næste dukker op.\\
For at holdet har ret til at løse jeres post, skal de være igang med jeres rute, og skal have løst de foregående ruter. I kan se om ruterne før jeres egen, har fået et hul på holdenes ID-kort.
\subsection{App og tjek ind}
På www.post25.sagalobet.dk skal I indberette holdenes point samt tjekke deltagerne ind og ud.\\
I skal logge ind med jeres postnummer - Altså \textbf{A4}\\
\newline
Kommer I til at lave en forkert registrering, skal I kontakte mål med det samme, på tlf. 3267 9499.
\newpage
\vspace*{.4cm}
\subsection{Postbeskrivelser}
I har fået udleveret 5 plader med postbeskrivelser til jeres post. På hver side af pladen er der en postbeskrivelse til enten væbner- eller seniorhold. I kan se på bjælken i toppen, om det er til væbner- eller seniorhold. Vi har i udgangspunktet ikke lavet andet end sproglige tilretninger på posten, men læs den lige igennem, så I er sikre på at vide det samme som deltagerne ved.\\
Vi vil \textbf{meget gerne} have pladerne retur i målområdet efter løbet.
\subsection{Sikkerhed på løbet}
For at give deltagerne en tryg oplevelse har vi nogle sikkerhedskrav til jeres post:
\begin{itemize}
  \item I skal opsætte jeres velkomstskilt, så det er det første deltagerne ser, når de kommer tæt på posten.
  \item I skal alle sammen bære de udleverede ID-kort yderst, så de er tydelige for deltagerne.
  \item I skal udlevere postbeskrivelserne til deltagerne, så de ved, at I er en ægte post.
  \item I skal huske at tjekke deltagerne ind og ud så snart de ankommer og forlader posten, så vi kan følge med i, hvor langt de er.
\end{itemize}
\subsection{Point}
Alle levende poster giver mellem 0 og 100 point, med mulighed for at optjene 25 bonuspoint, hvis posten løses på en bestemt måde. Giv kun deltagerne point ud fra deres opgaveløsning og ikke ud fra om de fedter for jer. Når et hold har gennemført posten, skal pointene indberettes i appen. Husk, at I ikke kan lave om i jeres registreringer, og det kun er muligt at indsende point én gang pr. hold.\\
\newline
Kommer I til at lave en forkert registrering, skal I kontakte mål med det samme, på tlf. 3267 9499.
\subsection{Åbningstider}
Alle poster åbner kl. 16:00 og lukker kl. 08:00.\\
Tjekker et hold ind på jeres post inden kl. 08.00, må de stadig gerne løse posten.\\
Hold, som ankommer efter kl. 08:00, kan ikke løse posten.\\
Basen lukker også kl. 08:00.
\subsection{Postforløb}
For at give deltagerne den samme oplevelse, vil vi gerne have jer til at følge dette postforløb:
\begin{itemize}
  \item Byd holdet velkommen til posten.
  \item Modtag holdets ID-kort og tjek dem ind på app\'en.
  \item Udlever postbeskrivelsen til holdet, hvis der ikke er kø, eller de er forrest i køen. Hvis holdet skal sidde i kø, skal I fortælle dem, hvor lang tid de kan forvente at vente.
  \item Sæt holdet i gang med at løse posten, når de har læst postbeskrivelsen.
  \item Fortæl holdet, hvor mange point de har tjent, når de er færdige med at løse posten.
  \item Registrer holdets point på app\'en.
  \item Tjek holdet ud på app\'en og giv dem deres ID-kort tilbage.
  \item Hold øje med om holdet er ved at gå kold. Giv dem noget opbakning, hvis det er tilfældet.
\end{itemize}
\newpage
\vspace*{.4cm}
\subsection{Baseområdet}
Hvis der er brug for at sende noget af jeres mandskab ind på en pause, kan de finde sig til rette i postområdet på basen. Husk at holdene kan komme hele natten, så sørg for ikke at sende hele postens mandskab afsted på en gang.\\
På basen er der en del af området, som er forbeholdt udvalget. Vi vil meget gerne hyggesnakke med jer og svare på spørgsmål, men vi har brug for ro i udvalgsområdet. Det håber, vi at I vil respektere. Har I brug for at nogen fra udvalget kommer ud, kan I sige det til de udvalgsmedlemmer, som bemander basebordet.
\subsection{Hvis en deltager vil udgå af løbet}
Undervejs på løbet kommer I til at opleve mange trætte børn og unge. Nogle af dem vil måske også gerne give op. Hvis det er tilfældet, skal I først spørge om de er sikre. Fortæl deltagerne, hvor langt der ca. er til næste post og hvem der bemander den. Det kan være, at det er deres egen kreds og at de gerne vil nå forbi den post også. Hvis deltageren stadig ønsker at give op, skal I finde et sted, de kan holde sig i læ og så ringe til mål på tlf. 3267 9499. Efterfølgende henter vi deltageren, når det er muligt. Deltagere som ønsker at give op, må gerne hjælpe sit hold med at løse posten.\\
\newline
Oplever I deltagere, hvor I selv tænker, at det vil være bedst for deltageren at udgå at løbet, så kontakt udvalget. Det er kun deltagerne eller udvalget der kan træffe beslutning om at udgå af løbet.
\subsection{Følg med i løbet}
I app\'en kan I se, hvor langt holdene er nået og se om der er nogen på vej til jeres post.
\subsection{Postpokalen}
Alle løbets levende poster bliver vurderet af deltagerne. Posterne får point på en skala fra 1-10.\\
Den samlede score udregnes som et gennemsnit af de givne point. For at vi får den bedst mulige konkurrence, må I meget gerne minde deltagerne om at give jer point til postpokalen, inden de tjekker ud.\\
Undgå også at fedte for deltagerne med slik, snacks eller andet. - I giver heller ikke deltagerne fedtepoint for at fedte for jer.\\
\newline
Udover deltagerne, vil forskellige jury-medlemmer dukke op i løbet af løbet for at vurdere jeres post. Vi håber at I vil tage godt imod dem.
\subsection{Hemmeligheder}
En stor del af de point, deltagerne kan opnå på Sagaløbet, findes som hemmelige og skjulte point. Opdager I sådan nogle, så lad være med at fortælle deltagerne om dem - også de deltagere fra jeres egen kreds. På den måde får vi den mest fair konkurrence.
\newpage
\vspace*{.4cm}
\subsection{Mærker}
Når vi er færdige med løbet, kan alt postmandskab, som ikke allerede har et mærke, få et SagaCrew-mærke.\\
Alle deltagere som gennemfører Sagaløbet får også et mærke. Deltagere som tidligere har gennemført løbet, men kun fået ét mærke, kan købe ekstra mærker efter præmieoverrækkelsen til 10 kr. stykket.
\subsection{Kontakt}
Får I brug for at komme i kontakt med udvalget gennem natten, så ring til mål på tlf. 3267 9499. Har I brug for at få fat på et bestemt udvalgsmedlem, kan vores kontaktoplysninger findes på hjemmesiden. Vi ser dog helst at I kontakter mål frem for at kontakte enkelte udvalgsmedlemmer, da vi ordner koordinationen herfra.
\newline
Til sidst er der ikke så meget andet at sige end tak!\begin{itemize}
  \item fordi I er med til at skabe Sagaløbet 2025
  \item for at være gode ambassadører for vores arrangement
  \item for at tage jeres væbnere, seniorvæbnere og seniorer med på løb
  \item for at sikre at Sagaløbet er en fair konkurrence
  \item for at give deltagerne uforglemmelige oplevelser, som giver dem mulighed for at vokse
  \item for at hjælpe os med at udvikle løbet
  \item for at give os muligheden for at skabe fantastiske FDF-oplevelser i fællesskab
  \item for alt det vi ikke får sagt tak for
\end{itemize}
Vi glæder os helt vildt til at besøge jer i løbet af natten, og til at opleve løbet sammen med jer og deltagerne!
\newline
Rigtig godt løb!\\
\newline
\textcolor{søblå}{De bedste hilsner}\\
\textcolor{natblå}{\textbf{Alberte, Augusta, Charlotte, Dan, Jens Peter, Jonas, Lasse, Lucas, Magnus \& Thor}}\\
\textcolor{natblå}{\textbf{Sagaløbsudvalget}}\\
\newpage
\title{Postinfo}\\
Kære \textbf{Vester Hassing}\\
\newline
Velkommen på Sagaløbet 2025.\\
Vi har glædet os vildt til at lave løb med jer.\\
\newline
I har fået \textbf{Post B1}. Posten er placeret på \textbf{RUTE B}\\
Vi har, som mange tidligere år, forsøgt at gøre løbet bedre for både deltagere og jer. I år drejer det sig om småjusteringer, så hvis I har været med før, så er det meste, som I kender det.\\
\newline
Hvis I ikke har været med før, så håber vi, at dette infobrev kan hjælpe jer med at være klædt godt på. Har I spørgsmål efter I har læst brevet, så sig til! Ring til mål på tlf. +45 3267 9499.
\subsection{Sagaløbets principper}
Til planlægningen af årets løb har vi skabt seks principper, som er det bærende element for både planlægning, afviklingen og udviklingen af Sagaløbet.\\
De seks principper er:
\begin{itemize}
  \item \textbf{Vi skaber Sagaløbet sammen}
  \item \textbf{Alle skal opleve at lykkes}
  \item \textbf{Alle skal opleve udfordringer}
  \item \textbf{Det kendte og ukendte}
  \item \textbf{Vi gør os umage}
  \item \textbf{Retfærdig konkurrence}
\end{itemize}
Vi håber, at I kan genkende at Sagaløbet er bygget op om disse seks bærende principper. Det bliver også med disse seks principper, at vi kommer til at træffe beslutninger under løbet. Har I spørgsmål til principperne, er vi altid klar til at tage en snak om dem.
\subsection{Ruter og deltagerstrøm}
Løbet er inddelt i fire mindre ruter. Hver rute er angivet med et bogstav og har 4 levende poster og én død post. Derudover er der 4 levende seniorposter i målområdet.\\
Ved at inddele løbet i fire mindre løb kan vi sprede deltagerne ud, have kontakt med dem løbende, give dem et helle og sikre et bedre flow i løbet.\\
\newline
Det betyder at I skal stå post fra 16:00 til 08:00.\\
Det er lang tid, og det kan være nødvendigt at hjælpe hinanden med at holde pauser undervejs.\\
\newline
Holdene er delt op i fire grupper med 12-13 hold i hver gruppe. Det betyder at I forhåbentlig når at se alle holdene fra en gruppe, før den næste dukker op.\\
For at holdet har ret til at løse jeres post, skal de være igang med jeres rute, og skal have løst de foregående ruter. I kan se om ruterne før jeres egen, har fået et hul på holdenes ID-kort.
\subsection{App og tjek ind}
På www.post25.sagalobet.dk skal I indberette holdenes point samt tjekke deltagerne ind og ud.\\
I skal logge ind med jeres postnummer - Altså \textbf{B1}\\
\newline
Kommer I til at lave en forkert registrering, skal I kontakte mål med det samme, på tlf. 3267 9499.
\newpage
\vspace*{.4cm}
\subsection{Postbeskrivelser}
I har fået udleveret 5 plader med postbeskrivelser til jeres post. På hver side af pladen er der en postbeskrivelse til enten væbner- eller seniorhold. I kan se på bjælken i toppen, om det er til væbner- eller seniorhold. Vi har i udgangspunktet ikke lavet andet end sproglige tilretninger på posten, men læs den lige igennem, så I er sikre på at vide det samme som deltagerne ved.\\
Vi vil \textbf{meget gerne} have pladerne retur i målområdet efter løbet.
\subsection{Sikkerhed på løbet}
For at give deltagerne en tryg oplevelse har vi nogle sikkerhedskrav til jeres post:
\begin{itemize}
  \item I skal opsætte jeres velkomstskilt, så det er det første deltagerne ser, når de kommer tæt på posten.
  \item I skal alle sammen bære de udleverede ID-kort yderst, så de er tydelige for deltagerne.
  \item I skal udlevere postbeskrivelserne til deltagerne, så de ved, at I er en ægte post.
  \item I skal huske at tjekke deltagerne ind og ud så snart de ankommer og forlader posten, så vi kan følge med i, hvor langt de er.
\end{itemize}
\subsection{Point}
Alle levende poster giver mellem 0 og 100 point, med mulighed for at optjene 25 bonuspoint, hvis posten løses på en bestemt måde. Giv kun deltagerne point ud fra deres opgaveløsning og ikke ud fra om de fedter for jer. Når et hold har gennemført posten, skal pointene indberettes i appen. Husk, at I ikke kan lave om i jeres registreringer, og det kun er muligt at indsende point én gang pr. hold.\\
\newline
Kommer I til at lave en forkert registrering, skal I kontakte mål med det samme, på tlf. 3267 9499.
\subsection{Åbningstider}
Alle poster åbner kl. 16:00 og lukker kl. 08:00.\\
Tjekker et hold ind på jeres post inden kl. 08.00, må de stadig gerne løse posten.\\
Hold, som ankommer efter kl. 08:00, kan ikke løse posten.\\
Basen lukker også kl. 08:00.
\subsection{Postforløb}
For at give deltagerne den samme oplevelse, vil vi gerne have jer til at følge dette postforløb:
\begin{itemize}
  \item Byd holdet velkommen til posten.
  \item Modtag holdets ID-kort og tjek dem ind på app\'en.
  \item Udlever postbeskrivelsen til holdet, hvis der ikke er kø, eller de er forrest i køen. Hvis holdet skal sidde i kø, skal I fortælle dem, hvor lang tid de kan forvente at vente.
  \item Sæt holdet i gang med at løse posten, når de har læst postbeskrivelsen.
  \item Fortæl holdet, hvor mange point de har tjent, når de er færdige med at løse posten.
  \item Registrer holdets point på app\'en.
  \item Tjek holdet ud på app\'en og giv dem deres ID-kort tilbage.
  \item Hold øje med om holdet er ved at gå kold. Giv dem noget opbakning, hvis det er tilfældet.
\end{itemize}
\newpage
\vspace*{.4cm}
\subsection{Baseområdet}
Hvis der er brug for at sende noget af jeres mandskab ind på en pause, kan de finde sig til rette i postområdet på basen. Husk at holdene kan komme hele natten, så sørg for ikke at sende hele postens mandskab afsted på en gang.\\
På basen er der en del af området, som er forbeholdt udvalget. Vi vil meget gerne hyggesnakke med jer og svare på spørgsmål, men vi har brug for ro i udvalgsområdet. Det håber, vi at I vil respektere. Har I brug for at nogen fra udvalget kommer ud, kan I sige det til de udvalgsmedlemmer, som bemander basebordet.
\subsection{Hvis en deltager vil udgå af løbet}
Undervejs på løbet kommer I til at opleve mange trætte børn og unge. Nogle af dem vil måske også gerne give op. Hvis det er tilfældet, skal I først spørge om de er sikre. Fortæl deltagerne, hvor langt der ca. er til næste post og hvem der bemander den. Det kan være, at det er deres egen kreds og at de gerne vil nå forbi den post også. Hvis deltageren stadig ønsker at give op, skal I finde et sted, de kan holde sig i læ og så ringe til mål på tlf. 3267 9499. Efterfølgende henter vi deltageren, når det er muligt. Deltagere som ønsker at give op, må gerne hjælpe sit hold med at løse posten.\\
\newline
Oplever I deltagere, hvor I selv tænker, at det vil være bedst for deltageren at udgå at løbet, så kontakt udvalget. Det er kun deltagerne eller udvalget der kan træffe beslutning om at udgå af løbet.
\subsection{Følg med i løbet}
I app\'en kan I se, hvor langt holdene er nået og se om der er nogen på vej til jeres post.
\subsection{Postpokalen}
Alle løbets levende poster bliver vurderet af deltagerne. Posterne får point på en skala fra 1-10.\\
Den samlede score udregnes som et gennemsnit af de givne point. For at vi får den bedst mulige konkurrence, må I meget gerne minde deltagerne om at give jer point til postpokalen, inden de tjekker ud.\\
Undgå også at fedte for deltagerne med slik, snacks eller andet. - I giver heller ikke deltagerne fedtepoint for at fedte for jer.\\
\newline
Udover deltagerne, vil forskellige jury-medlemmer dukke op i løbet af løbet for at vurdere jeres post. Vi håber at I vil tage godt imod dem.
\subsection{Hemmeligheder}
En stor del af de point, deltagerne kan opnå på Sagaløbet, findes som hemmelige og skjulte point. Opdager I sådan nogle, så lad være med at fortælle deltagerne om dem - også de deltagere fra jeres egen kreds. På den måde får vi den mest fair konkurrence.
\newpage
\vspace*{.4cm}
\subsection{Mærker}
Når vi er færdige med løbet, kan alt postmandskab, som ikke allerede har et mærke, få et SagaCrew-mærke.\\
Alle deltagere som gennemfører Sagaløbet får også et mærke. Deltagere som tidligere har gennemført løbet, men kun fået ét mærke, kan købe ekstra mærker efter præmieoverrækkelsen til 10 kr. stykket.
\subsection{Kontakt}
Får I brug for at komme i kontakt med udvalget gennem natten, så ring til mål på tlf. 3267 9499. Har I brug for at få fat på et bestemt udvalgsmedlem, kan vores kontaktoplysninger findes på hjemmesiden. Vi ser dog helst at I kontakter mål frem for at kontakte enkelte udvalgsmedlemmer, da vi ordner koordinationen herfra.
\newline
Til sidst er der ikke så meget andet at sige end tak!\begin{itemize}
  \item fordi I er med til at skabe Sagaløbet 2025
  \item for at være gode ambassadører for vores arrangement
  \item for at tage jeres væbnere, seniorvæbnere og seniorer med på løb
  \item for at sikre at Sagaløbet er en fair konkurrence
  \item for at give deltagerne uforglemmelige oplevelser, som giver dem mulighed for at vokse
  \item for at hjælpe os med at udvikle løbet
  \item for at give os muligheden for at skabe fantastiske FDF-oplevelser i fællesskab
  \item for alt det vi ikke får sagt tak for
\end{itemize}
Vi glæder os helt vildt til at besøge jer i løbet af natten, og til at opleve løbet sammen med jer og deltagerne!
\newline
Rigtig godt løb!\\
\newline
\textcolor{søblå}{De bedste hilsner}\\
\textcolor{natblå}{\textbf{Alberte, Augusta, Charlotte, Dan, Jens Peter, Jonas, Lasse, Lucas, Magnus \& Thor}}\\
\textcolor{natblå}{\textbf{Sagaløbsudvalget}}\\
\newpage
\title{Postinfo}\\
Kære \textbf{Vodskov}\\
\newline
Velkommen på Sagaløbet 2025.\\
Vi har glædet os vildt til at lave løb med jer.\\
\newline
I har fået \textbf{Post B2}. Posten er placeret på \textbf{RUTE B}\\
Vi har, som mange tidligere år, forsøgt at gøre løbet bedre for både deltagere og jer. I år drejer det sig om småjusteringer, så hvis I har været med før, så er det meste, som I kender det.\\
\newline
Hvis I ikke har været med før, så håber vi, at dette infobrev kan hjælpe jer med at være klædt godt på. Har I spørgsmål efter I har læst brevet, så sig til! Ring til mål på tlf. +45 3267 9499.
\subsection{Sagaløbets principper}
Til planlægningen af årets løb har vi skabt seks principper, som er det bærende element for både planlægning, afviklingen og udviklingen af Sagaløbet.\\
De seks principper er:
\begin{itemize}
  \item \textbf{Vi skaber Sagaløbet sammen}
  \item \textbf{Alle skal opleve at lykkes}
  \item \textbf{Alle skal opleve udfordringer}
  \item \textbf{Det kendte og ukendte}
  \item \textbf{Vi gør os umage}
  \item \textbf{Retfærdig konkurrence}
\end{itemize}
Vi håber, at I kan genkende at Sagaløbet er bygget op om disse seks bærende principper. Det bliver også med disse seks principper, at vi kommer til at træffe beslutninger under løbet. Har I spørgsmål til principperne, er vi altid klar til at tage en snak om dem.
\subsection{Ruter og deltagerstrøm}
Løbet er inddelt i fire mindre ruter. Hver rute er angivet med et bogstav og har 4 levende poster og én død post. Derudover er der 4 levende seniorposter i målområdet.\\
Ved at inddele løbet i fire mindre løb kan vi sprede deltagerne ud, have kontakt med dem løbende, give dem et helle og sikre et bedre flow i løbet.\\
\newline
Det betyder at I skal stå post fra 16:00 til 08:00.\\
Det er lang tid, og det kan være nødvendigt at hjælpe hinanden med at holde pauser undervejs.\\
\newline
Holdene er delt op i fire grupper med 12-13 hold i hver gruppe. Det betyder at I forhåbentlig når at se alle holdene fra en gruppe, før den næste dukker op.\\
For at holdet har ret til at løse jeres post, skal de være igang med jeres rute, og skal have løst de foregående ruter. I kan se om ruterne før jeres egen, har fået et hul på holdenes ID-kort.
\subsection{App og tjek ind}
På www.post25.sagalobet.dk skal I indberette holdenes point samt tjekke deltagerne ind og ud.\\
I skal logge ind med jeres postnummer - Altså \textbf{B2}\\
\newline
Kommer I til at lave en forkert registrering, skal I kontakte mål med det samme, på tlf. 3267 9499.
\newpage
\vspace*{.4cm}
\subsection{Postbeskrivelser}
I har fået udleveret 5 plader med postbeskrivelser til jeres post. På hver side af pladen er der en postbeskrivelse til enten væbner- eller seniorhold. I kan se på bjælken i toppen, om det er til væbner- eller seniorhold. Vi har i udgangspunktet ikke lavet andet end sproglige tilretninger på posten, men læs den lige igennem, så I er sikre på at vide det samme som deltagerne ved.\\
Vi vil \textbf{meget gerne} have pladerne retur i målområdet efter løbet.
\subsection{Sikkerhed på løbet}
For at give deltagerne en tryg oplevelse har vi nogle sikkerhedskrav til jeres post:
\begin{itemize}
  \item I skal opsætte jeres velkomstskilt, så det er det første deltagerne ser, når de kommer tæt på posten.
  \item I skal alle sammen bære de udleverede ID-kort yderst, så de er tydelige for deltagerne.
  \item I skal udlevere postbeskrivelserne til deltagerne, så de ved, at I er en ægte post.
  \item I skal huske at tjekke deltagerne ind og ud så snart de ankommer og forlader posten, så vi kan følge med i, hvor langt de er.
\end{itemize}
\subsection{Point}
Alle levende poster giver mellem 0 og 100 point, med mulighed for at optjene 25 bonuspoint, hvis posten løses på en bestemt måde. Giv kun deltagerne point ud fra deres opgaveløsning og ikke ud fra om de fedter for jer. Når et hold har gennemført posten, skal pointene indberettes i appen. Husk, at I ikke kan lave om i jeres registreringer, og det kun er muligt at indsende point én gang pr. hold.\\
\newline
Kommer I til at lave en forkert registrering, skal I kontakte mål med det samme, på tlf. 3267 9499.
\subsection{Åbningstider}
Alle poster åbner kl. 16:00 og lukker kl. 08:00.\\
Tjekker et hold ind på jeres post inden kl. 08.00, må de stadig gerne løse posten.\\
Hold, som ankommer efter kl. 08:00, kan ikke løse posten.\\
Basen lukker også kl. 08:00.
\subsection{Postforløb}
For at give deltagerne den samme oplevelse, vil vi gerne have jer til at følge dette postforløb:
\begin{itemize}
  \item Byd holdet velkommen til posten.
  \item Modtag holdets ID-kort og tjek dem ind på app\'en.
  \item Udlever postbeskrivelsen til holdet, hvis der ikke er kø, eller de er forrest i køen. Hvis holdet skal sidde i kø, skal I fortælle dem, hvor lang tid de kan forvente at vente.
  \item Sæt holdet i gang med at løse posten, når de har læst postbeskrivelsen.
  \item Fortæl holdet, hvor mange point de har tjent, når de er færdige med at løse posten.
  \item Registrer holdets point på app\'en.
  \item Tjek holdet ud på app\'en og giv dem deres ID-kort tilbage.
  \item Hold øje med om holdet er ved at gå kold. Giv dem noget opbakning, hvis det er tilfældet.
\end{itemize}
\newpage
\vspace*{.4cm}
\subsection{Baseområdet}
Hvis der er brug for at sende noget af jeres mandskab ind på en pause, kan de finde sig til rette i postområdet på basen. Husk at holdene kan komme hele natten, så sørg for ikke at sende hele postens mandskab afsted på en gang.\\
På basen er der en del af området, som er forbeholdt udvalget. Vi vil meget gerne hyggesnakke med jer og svare på spørgsmål, men vi har brug for ro i udvalgsområdet. Det håber, vi at I vil respektere. Har I brug for at nogen fra udvalget kommer ud, kan I sige det til de udvalgsmedlemmer, som bemander basebordet.
\subsection{Hvis en deltager vil udgå af løbet}
Undervejs på løbet kommer I til at opleve mange trætte børn og unge. Nogle af dem vil måske også gerne give op. Hvis det er tilfældet, skal I først spørge om de er sikre. Fortæl deltagerne, hvor langt der ca. er til næste post og hvem der bemander den. Det kan være, at det er deres egen kreds og at de gerne vil nå forbi den post også. Hvis deltageren stadig ønsker at give op, skal I finde et sted, de kan holde sig i læ og så ringe til mål på tlf. 3267 9499. Efterfølgende henter vi deltageren, når det er muligt. Deltagere som ønsker at give op, må gerne hjælpe sit hold med at løse posten.\\
\newline
Oplever I deltagere, hvor I selv tænker, at det vil være bedst for deltageren at udgå at løbet, så kontakt udvalget. Det er kun deltagerne eller udvalget der kan træffe beslutning om at udgå af løbet.
\subsection{Følg med i løbet}
I app\'en kan I se, hvor langt holdene er nået og se om der er nogen på vej til jeres post.
\subsection{Postpokalen}
Alle løbets levende poster bliver vurderet af deltagerne. Posterne får point på en skala fra 1-10.\\
Den samlede score udregnes som et gennemsnit af de givne point. For at vi får den bedst mulige konkurrence, må I meget gerne minde deltagerne om at give jer point til postpokalen, inden de tjekker ud.\\
Undgå også at fedte for deltagerne med slik, snacks eller andet. - I giver heller ikke deltagerne fedtepoint for at fedte for jer.\\
\newline
Udover deltagerne, vil forskellige jury-medlemmer dukke op i løbet af løbet for at vurdere jeres post. Vi håber at I vil tage godt imod dem.
\subsection{Hemmeligheder}
En stor del af de point, deltagerne kan opnå på Sagaløbet, findes som hemmelige og skjulte point. Opdager I sådan nogle, så lad være med at fortælle deltagerne om dem - også de deltagere fra jeres egen kreds. På den måde får vi den mest fair konkurrence.
\newpage
\vspace*{.4cm}
\subsection{Mærker}
Når vi er færdige med løbet, kan alt postmandskab, som ikke allerede har et mærke, få et SagaCrew-mærke.\\
Alle deltagere som gennemfører Sagaløbet får også et mærke. Deltagere som tidligere har gennemført løbet, men kun fået ét mærke, kan købe ekstra mærker efter præmieoverrækkelsen til 10 kr. stykket.
\subsection{Kontakt}
Får I brug for at komme i kontakt med udvalget gennem natten, så ring til mål på tlf. 3267 9499. Har I brug for at få fat på et bestemt udvalgsmedlem, kan vores kontaktoplysninger findes på hjemmesiden. Vi ser dog helst at I kontakter mål frem for at kontakte enkelte udvalgsmedlemmer, da vi ordner koordinationen herfra.
\newline
Til sidst er der ikke så meget andet at sige end tak!\begin{itemize}
  \item fordi I er med til at skabe Sagaløbet 2025
  \item for at være gode ambassadører for vores arrangement
  \item for at tage jeres væbnere, seniorvæbnere og seniorer med på løb
  \item for at sikre at Sagaløbet er en fair konkurrence
  \item for at give deltagerne uforglemmelige oplevelser, som giver dem mulighed for at vokse
  \item for at hjælpe os med at udvikle løbet
  \item for at give os muligheden for at skabe fantastiske FDF-oplevelser i fællesskab
  \item for alt det vi ikke får sagt tak for
\end{itemize}
Vi glæder os helt vildt til at besøge jer i løbet af natten, og til at opleve løbet sammen med jer og deltagerne!
\newline
Rigtig godt løb!\\
\newline
\textcolor{søblå}{De bedste hilsner}\\
\textcolor{natblå}{\textbf{Alberte, Augusta, Charlotte, Dan, Jens Peter, Jonas, Lasse, Lucas, Magnus \& Thor}}\\
\textcolor{natblå}{\textbf{Sagaløbsudvalget}}\\
\newpage
\title{Postinfo}\\
Kære \textbf{Skalborg}\\
\newline
Velkommen på Sagaløbet 2025.\\
Vi har glædet os vildt til at lave løb med jer.\\
\newline
I har fået \textbf{Post B3}. Posten er placeret på \textbf{RUTE B}\\
Vi har, som mange tidligere år, forsøgt at gøre løbet bedre for både deltagere og jer. I år drejer det sig om småjusteringer, så hvis I har været med før, så er det meste, som I kender det.\\
\newline
Hvis I ikke har været med før, så håber vi, at dette infobrev kan hjælpe jer med at være klædt godt på. Har I spørgsmål efter I har læst brevet, så sig til! Ring til mål på tlf. +45 3267 9499.
\subsection{Sagaløbets principper}
Til planlægningen af årets løb har vi skabt seks principper, som er det bærende element for både planlægning, afviklingen og udviklingen af Sagaløbet.\\
De seks principper er:
\begin{itemize}
  \item \textbf{Vi skaber Sagaløbet sammen}
  \item \textbf{Alle skal opleve at lykkes}
  \item \textbf{Alle skal opleve udfordringer}
  \item \textbf{Det kendte og ukendte}
  \item \textbf{Vi gør os umage}
  \item \textbf{Retfærdig konkurrence}
\end{itemize}
Vi håber, at I kan genkende at Sagaløbet er bygget op om disse seks bærende principper. Det bliver også med disse seks principper, at vi kommer til at træffe beslutninger under løbet. Har I spørgsmål til principperne, er vi altid klar til at tage en snak om dem.
\subsection{Ruter og deltagerstrøm}
Løbet er inddelt i fire mindre ruter. Hver rute er angivet med et bogstav og har 4 levende poster og én død post. Derudover er der 4 levende seniorposter i målområdet.\\
Ved at inddele løbet i fire mindre løb kan vi sprede deltagerne ud, have kontakt med dem løbende, give dem et helle og sikre et bedre flow i løbet.\\
\newline
Det betyder at I skal stå post fra 16:00 til 08:00.\\
Det er lang tid, og det kan være nødvendigt at hjælpe hinanden med at holde pauser undervejs.\\
\newline
Holdene er delt op i fire grupper med 12-13 hold i hver gruppe. Det betyder at I forhåbentlig når at se alle holdene fra en gruppe, før den næste dukker op.\\
For at holdet har ret til at løse jeres post, skal de være igang med jeres rute, og skal have løst de foregående ruter. I kan se om ruterne før jeres egen, har fået et hul på holdenes ID-kort.
\subsection{App og tjek ind}
På www.post25.sagalobet.dk skal I indberette holdenes point samt tjekke deltagerne ind og ud.\\
I skal logge ind med jeres postnummer - Altså \textbf{B3}\\
\newline
Kommer I til at lave en forkert registrering, skal I kontakte mål med det samme, på tlf. 3267 9499.
\newpage
\vspace*{.4cm}
\subsection{Postbeskrivelser}
I har fået udleveret 5 plader med postbeskrivelser til jeres post. På hver side af pladen er der en postbeskrivelse til enten væbner- eller seniorhold. I kan se på bjælken i toppen, om det er til væbner- eller seniorhold. Vi har i udgangspunktet ikke lavet andet end sproglige tilretninger på posten, men læs den lige igennem, så I er sikre på at vide det samme som deltagerne ved.\\
Vi vil \textbf{meget gerne} have pladerne retur i målområdet efter løbet.
\subsection{Sikkerhed på løbet}
For at give deltagerne en tryg oplevelse har vi nogle sikkerhedskrav til jeres post:
\begin{itemize}
  \item I skal opsætte jeres velkomstskilt, så det er det første deltagerne ser, når de kommer tæt på posten.
  \item I skal alle sammen bære de udleverede ID-kort yderst, så de er tydelige for deltagerne.
  \item I skal udlevere postbeskrivelserne til deltagerne, så de ved, at I er en ægte post.
  \item I skal huske at tjekke deltagerne ind og ud så snart de ankommer og forlader posten, så vi kan følge med i, hvor langt de er.
\end{itemize}
\subsection{Point}
Alle levende poster giver mellem 0 og 100 point, med mulighed for at optjene 25 bonuspoint, hvis posten løses på en bestemt måde. Giv kun deltagerne point ud fra deres opgaveløsning og ikke ud fra om de fedter for jer. Når et hold har gennemført posten, skal pointene indberettes i appen. Husk, at I ikke kan lave om i jeres registreringer, og det kun er muligt at indsende point én gang pr. hold.\\
\newline
Kommer I til at lave en forkert registrering, skal I kontakte mål med det samme, på tlf. 3267 9499.
\subsection{Åbningstider}
Alle poster åbner kl. 16:00 og lukker kl. 08:00.\\
Tjekker et hold ind på jeres post inden kl. 08.00, må de stadig gerne løse posten.\\
Hold, som ankommer efter kl. 08:00, kan ikke løse posten.\\
Basen lukker også kl. 08:00.
\subsection{Postforløb}
For at give deltagerne den samme oplevelse, vil vi gerne have jer til at følge dette postforløb:
\begin{itemize}
  \item Byd holdet velkommen til posten.
  \item Modtag holdets ID-kort og tjek dem ind på app\'en.
  \item Udlever postbeskrivelsen til holdet, hvis der ikke er kø, eller de er forrest i køen. Hvis holdet skal sidde i kø, skal I fortælle dem, hvor lang tid de kan forvente at vente.
  \item Sæt holdet i gang med at løse posten, når de har læst postbeskrivelsen.
  \item Fortæl holdet, hvor mange point de har tjent, når de er færdige med at løse posten.
  \item Registrer holdets point på app\'en.
  \item Tjek holdet ud på app\'en og giv dem deres ID-kort tilbage.
  \item Hold øje med om holdet er ved at gå kold. Giv dem noget opbakning, hvis det er tilfældet.
\end{itemize}
\newpage
\vspace*{.4cm}
\subsection{Baseområdet}
Hvis der er brug for at sende noget af jeres mandskab ind på en pause, kan de finde sig til rette i postområdet på basen. Husk at holdene kan komme hele natten, så sørg for ikke at sende hele postens mandskab afsted på en gang.\\
På basen er der en del af området, som er forbeholdt udvalget. Vi vil meget gerne hyggesnakke med jer og svare på spørgsmål, men vi har brug for ro i udvalgsområdet. Det håber, vi at I vil respektere. Har I brug for at nogen fra udvalget kommer ud, kan I sige det til de udvalgsmedlemmer, som bemander basebordet.
\subsection{Hvis en deltager vil udgå af løbet}
Undervejs på løbet kommer I til at opleve mange trætte børn og unge. Nogle af dem vil måske også gerne give op. Hvis det er tilfældet, skal I først spørge om de er sikre. Fortæl deltagerne, hvor langt der ca. er til næste post og hvem der bemander den. Det kan være, at det er deres egen kreds og at de gerne vil nå forbi den post også. Hvis deltageren stadig ønsker at give op, skal I finde et sted, de kan holde sig i læ og så ringe til mål på tlf. 3267 9499. Efterfølgende henter vi deltageren, når det er muligt. Deltagere som ønsker at give op, må gerne hjælpe sit hold med at løse posten.\\
\newline
Oplever I deltagere, hvor I selv tænker, at det vil være bedst for deltageren at udgå at løbet, så kontakt udvalget. Det er kun deltagerne eller udvalget der kan træffe beslutning om at udgå af løbet.
\subsection{Følg med i løbet}
I app\'en kan I se, hvor langt holdene er nået og se om der er nogen på vej til jeres post.
\subsection{Postpokalen}
Alle løbets levende poster bliver vurderet af deltagerne. Posterne får point på en skala fra 1-10.\\
Den samlede score udregnes som et gennemsnit af de givne point. For at vi får den bedst mulige konkurrence, må I meget gerne minde deltagerne om at give jer point til postpokalen, inden de tjekker ud.\\
Undgå også at fedte for deltagerne med slik, snacks eller andet. - I giver heller ikke deltagerne fedtepoint for at fedte for jer.\\
\newline
Udover deltagerne, vil forskellige jury-medlemmer dukke op i løbet af løbet for at vurdere jeres post. Vi håber at I vil tage godt imod dem.
\subsection{Hemmeligheder}
En stor del af de point, deltagerne kan opnå på Sagaløbet, findes som hemmelige og skjulte point. Opdager I sådan nogle, så lad være med at fortælle deltagerne om dem - også de deltagere fra jeres egen kreds. På den måde får vi den mest fair konkurrence.
\newpage
\vspace*{.4cm}
\subsection{Mærker}
Når vi er færdige med løbet, kan alt postmandskab, som ikke allerede har et mærke, få et SagaCrew-mærke.\\
Alle deltagere som gennemfører Sagaløbet får også et mærke. Deltagere som tidligere har gennemført løbet, men kun fået ét mærke, kan købe ekstra mærker efter præmieoverrækkelsen til 10 kr. stykket.
\subsection{Kontakt}
Får I brug for at komme i kontakt med udvalget gennem natten, så ring til mål på tlf. 3267 9499. Har I brug for at få fat på et bestemt udvalgsmedlem, kan vores kontaktoplysninger findes på hjemmesiden. Vi ser dog helst at I kontakter mål frem for at kontakte enkelte udvalgsmedlemmer, da vi ordner koordinationen herfra.
\newline
Til sidst er der ikke så meget andet at sige end tak!\begin{itemize}
  \item fordi I er med til at skabe Sagaløbet 2025
  \item for at være gode ambassadører for vores arrangement
  \item for at tage jeres væbnere, seniorvæbnere og seniorer med på løb
  \item for at sikre at Sagaløbet er en fair konkurrence
  \item for at give deltagerne uforglemmelige oplevelser, som giver dem mulighed for at vokse
  \item for at hjælpe os med at udvikle løbet
  \item for at give os muligheden for at skabe fantastiske FDF-oplevelser i fællesskab
  \item for alt det vi ikke får sagt tak for
\end{itemize}
Vi glæder os helt vildt til at besøge jer i løbet af natten, og til at opleve løbet sammen med jer og deltagerne!
\newline
Rigtig godt løb!\\
\newline
\textcolor{søblå}{De bedste hilsner}\\
\textcolor{natblå}{\textbf{Alberte, Augusta, Charlotte, Dan, Jens Peter, Jonas, Lasse, Lucas, Magnus \& Thor}}\\
\textcolor{natblå}{\textbf{Sagaløbsudvalget}}\\
\newpage
\title{Postinfo}\\
Kære \textbf{Nørre Tranders}\\
\newline
Velkommen på Sagaløbet 2025.\\
Vi har glædet os vildt til at lave løb med jer.\\
\newline
I har fået \textbf{Post C1}. Posten er placeret på \textbf{RUTE C}\\
Vi har, som mange tidligere år, forsøgt at gøre løbet bedre for både deltagere og jer. I år drejer det sig om småjusteringer, så hvis I har været med før, så er det meste, som I kender det.\\
\newline
Hvis I ikke har været med før, så håber vi, at dette infobrev kan hjælpe jer med at være klædt godt på. Har I spørgsmål efter I har læst brevet, så sig til! Ring til mål på tlf. +45 3267 9499.
\subsection{Sagaløbets principper}
Til planlægningen af årets løb har vi skabt seks principper, som er det bærende element for både planlægning, afviklingen og udviklingen af Sagaløbet.\\
De seks principper er:
\begin{itemize}
  \item \textbf{Vi skaber Sagaløbet sammen}
  \item \textbf{Alle skal opleve at lykkes}
  \item \textbf{Alle skal opleve udfordringer}
  \item \textbf{Det kendte og ukendte}
  \item \textbf{Vi gør os umage}
  \item \textbf{Retfærdig konkurrence}
\end{itemize}
Vi håber, at I kan genkende at Sagaløbet er bygget op om disse seks bærende principper. Det bliver også med disse seks principper, at vi kommer til at træffe beslutninger under løbet. Har I spørgsmål til principperne, er vi altid klar til at tage en snak om dem.
\subsection{Ruter og deltagerstrøm}
Løbet er inddelt i fire mindre ruter. Hver rute er angivet med et bogstav og har 4 levende poster og én død post. Derudover er der 4 levende seniorposter i målområdet.\\
Ved at inddele løbet i fire mindre løb kan vi sprede deltagerne ud, have kontakt med dem løbende, give dem et helle og sikre et bedre flow i løbet.\\
\newline
Det betyder at I skal stå post fra 16:00 til 08:00.\\
Det er lang tid, og det kan være nødvendigt at hjælpe hinanden med at holde pauser undervejs.\\
\newline
Holdene er delt op i fire grupper med 12-13 hold i hver gruppe. Det betyder at I forhåbentlig når at se alle holdene fra en gruppe, før den næste dukker op.\\
For at holdet har ret til at løse jeres post, skal de være igang med jeres rute, og skal have løst de foregående ruter. I kan se om ruterne før jeres egen, har fået et hul på holdenes ID-kort.
\subsection{App og tjek ind}
På www.post25.sagalobet.dk skal I indberette holdenes point samt tjekke deltagerne ind og ud.\\
I skal logge ind med jeres postnummer - Altså \textbf{C1}\\
\newline
Kommer I til at lave en forkert registrering, skal I kontakte mål med det samme, på tlf. 3267 9499.
\newpage
\vspace*{.4cm}
\subsection{Postbeskrivelser}
I har fået udleveret 5 plader med postbeskrivelser til jeres post. På hver side af pladen er der en postbeskrivelse til enten væbner- eller seniorhold. I kan se på bjælken i toppen, om det er til væbner- eller seniorhold. Vi har i udgangspunktet ikke lavet andet end sproglige tilretninger på posten, men læs den lige igennem, så I er sikre på at vide det samme som deltagerne ved.\\
Vi vil \textbf{meget gerne} have pladerne retur i målområdet efter løbet.
\subsection{Sikkerhed på løbet}
For at give deltagerne en tryg oplevelse har vi nogle sikkerhedskrav til jeres post:
\begin{itemize}
  \item I skal opsætte jeres velkomstskilt, så det er det første deltagerne ser, når de kommer tæt på posten.
  \item I skal alle sammen bære de udleverede ID-kort yderst, så de er tydelige for deltagerne.
  \item I skal udlevere postbeskrivelserne til deltagerne, så de ved, at I er en ægte post.
  \item I skal huske at tjekke deltagerne ind og ud så snart de ankommer og forlader posten, så vi kan følge med i, hvor langt de er.
\end{itemize}
\subsection{Point}
Alle levende poster giver mellem 0 og 100 point, med mulighed for at optjene 25 bonuspoint, hvis posten løses på en bestemt måde. Giv kun deltagerne point ud fra deres opgaveløsning og ikke ud fra om de fedter for jer. Når et hold har gennemført posten, skal pointene indberettes i appen. Husk, at I ikke kan lave om i jeres registreringer, og det kun er muligt at indsende point én gang pr. hold.\\
\newline
Kommer I til at lave en forkert registrering, skal I kontakte mål med det samme, på tlf. 3267 9499.
\subsection{Åbningstider}
Alle poster åbner kl. 16:00 og lukker kl. 08:00.\\
Tjekker et hold ind på jeres post inden kl. 08.00, må de stadig gerne løse posten.\\
Hold, som ankommer efter kl. 08:00, kan ikke løse posten.\\
Basen lukker også kl. 08:00.
\subsection{Postforløb}
For at give deltagerne den samme oplevelse, vil vi gerne have jer til at følge dette postforløb:
\begin{itemize}
  \item Byd holdet velkommen til posten.
  \item Modtag holdets ID-kort og tjek dem ind på app\'en.
  \item Udlever postbeskrivelsen til holdet, hvis der ikke er kø, eller de er forrest i køen. Hvis holdet skal sidde i kø, skal I fortælle dem, hvor lang tid de kan forvente at vente.
  \item Sæt holdet i gang med at løse posten, når de har læst postbeskrivelsen.
  \item Fortæl holdet, hvor mange point de har tjent, når de er færdige med at løse posten.
  \item Registrer holdets point på app\'en.
  \item Tjek holdet ud på app\'en og giv dem deres ID-kort tilbage.
  \item Hold øje med om holdet er ved at gå kold. Giv dem noget opbakning, hvis det er tilfældet.
\end{itemize}
\newpage
\vspace*{.4cm}
\subsection{Baseområdet}
Hvis der er brug for at sende noget af jeres mandskab ind på en pause, kan de finde sig til rette i postområdet på basen. Husk at holdene kan komme hele natten, så sørg for ikke at sende hele postens mandskab afsted på en gang.\\
På basen er der en del af området, som er forbeholdt udvalget. Vi vil meget gerne hyggesnakke med jer og svare på spørgsmål, men vi har brug for ro i udvalgsområdet. Det håber, vi at I vil respektere. Har I brug for at nogen fra udvalget kommer ud, kan I sige det til de udvalgsmedlemmer, som bemander basebordet.
\subsection{Hvis en deltager vil udgå af løbet}
Undervejs på løbet kommer I til at opleve mange trætte børn og unge. Nogle af dem vil måske også gerne give op. Hvis det er tilfældet, skal I først spørge om de er sikre. Fortæl deltagerne, hvor langt der ca. er til næste post og hvem der bemander den. Det kan være, at det er deres egen kreds og at de gerne vil nå forbi den post også. Hvis deltageren stadig ønsker at give op, skal I finde et sted, de kan holde sig i læ og så ringe til mål på tlf. 3267 9499. Efterfølgende henter vi deltageren, når det er muligt. Deltagere som ønsker at give op, må gerne hjælpe sit hold med at løse posten.\\
\newline
Oplever I deltagere, hvor I selv tænker, at det vil være bedst for deltageren at udgå at løbet, så kontakt udvalget. Det er kun deltagerne eller udvalget der kan træffe beslutning om at udgå af løbet.
\subsection{Følg med i løbet}
I app\'en kan I se, hvor langt holdene er nået og se om der er nogen på vej til jeres post.
\subsection{Postpokalen}
Alle løbets levende poster bliver vurderet af deltagerne. Posterne får point på en skala fra 1-10.\\
Den samlede score udregnes som et gennemsnit af de givne point. For at vi får den bedst mulige konkurrence, må I meget gerne minde deltagerne om at give jer point til postpokalen, inden de tjekker ud.\\
Undgå også at fedte for deltagerne med slik, snacks eller andet. - I giver heller ikke deltagerne fedtepoint for at fedte for jer.\\
\newline
Udover deltagerne, vil forskellige jury-medlemmer dukke op i løbet af løbet for at vurdere jeres post. Vi håber at I vil tage godt imod dem.
\subsection{Hemmeligheder}
En stor del af de point, deltagerne kan opnå på Sagaløbet, findes som hemmelige og skjulte point. Opdager I sådan nogle, så lad være med at fortælle deltagerne om dem - også de deltagere fra jeres egen kreds. På den måde får vi den mest fair konkurrence.
\newpage
\vspace*{.4cm}
\subsection{Mærker}
Når vi er færdige med løbet, kan alt postmandskab, som ikke allerede har et mærke, få et SagaCrew-mærke.\\
Alle deltagere som gennemfører Sagaløbet får også et mærke. Deltagere som tidligere har gennemført løbet, men kun fået ét mærke, kan købe ekstra mærker efter præmieoverrækkelsen til 10 kr. stykket.
\subsection{Kontakt}
Får I brug for at komme i kontakt med udvalget gennem natten, så ring til mål på tlf. 3267 9499. Har I brug for at få fat på et bestemt udvalgsmedlem, kan vores kontaktoplysninger findes på hjemmesiden. Vi ser dog helst at I kontakter mål frem for at kontakte enkelte udvalgsmedlemmer, da vi ordner koordinationen herfra.
\newline
Til sidst er der ikke så meget andet at sige end tak!\begin{itemize}
  \item fordi I er med til at skabe Sagaløbet 2025
  \item for at være gode ambassadører for vores arrangement
  \item for at tage jeres væbnere, seniorvæbnere og seniorer med på løb
  \item for at sikre at Sagaløbet er en fair konkurrence
  \item for at give deltagerne uforglemmelige oplevelser, som giver dem mulighed for at vokse
  \item for at hjælpe os med at udvikle løbet
  \item for at give os muligheden for at skabe fantastiske FDF-oplevelser i fællesskab
  \item for alt det vi ikke får sagt tak for
\end{itemize}
Vi glæder os helt vildt til at besøge jer i løbet af natten, og til at opleve løbet sammen med jer og deltagerne!
\newline
Rigtig godt løb!\\
\newline
\textcolor{søblå}{De bedste hilsner}\\
\textcolor{natblå}{\textbf{Alberte, Augusta, Charlotte, Dan, Jens Peter, Jonas, Lasse, Lucas, Magnus \& Thor}}\\
\textcolor{natblå}{\textbf{Sagaløbsudvalget}}\\
\newpage
\title{Postinfo}\\
Kære \textbf{Sønderholm-Frejlev}\\
\newline
Velkommen på Sagaløbet 2025.\\
Vi har glædet os vildt til at lave løb med jer.\\
\newline
I har fået \textbf{Post C2}. Posten er placeret på \textbf{RUTE C}\\
Vi har, som mange tidligere år, forsøgt at gøre løbet bedre for både deltagere og jer. I år drejer det sig om småjusteringer, så hvis I har været med før, så er det meste, som I kender det.\\
\newline
Hvis I ikke har været med før, så håber vi, at dette infobrev kan hjælpe jer med at være klædt godt på. Har I spørgsmål efter I har læst brevet, så sig til! Ring til mål på tlf. +45 3267 9499.
\subsection{Sagaløbets principper}
Til planlægningen af årets løb har vi skabt seks principper, som er det bærende element for både planlægning, afviklingen og udviklingen af Sagaløbet.\\
De seks principper er:
\begin{itemize}
  \item \textbf{Vi skaber Sagaløbet sammen}
  \item \textbf{Alle skal opleve at lykkes}
  \item \textbf{Alle skal opleve udfordringer}
  \item \textbf{Det kendte og ukendte}
  \item \textbf{Vi gør os umage}
  \item \textbf{Retfærdig konkurrence}
\end{itemize}
Vi håber, at I kan genkende at Sagaløbet er bygget op om disse seks bærende principper. Det bliver også med disse seks principper, at vi kommer til at træffe beslutninger under løbet. Har I spørgsmål til principperne, er vi altid klar til at tage en snak om dem.
\subsection{Ruter og deltagerstrøm}
Løbet er inddelt i fire mindre ruter. Hver rute er angivet med et bogstav og har 4 levende poster og én død post. Derudover er der 4 levende seniorposter i målområdet.\\
Ved at inddele løbet i fire mindre løb kan vi sprede deltagerne ud, have kontakt med dem løbende, give dem et helle og sikre et bedre flow i løbet.\\
\newline
Det betyder at I skal stå post fra 16:00 til 08:00.\\
Det er lang tid, og det kan være nødvendigt at hjælpe hinanden med at holde pauser undervejs.\\
\newline
Holdene er delt op i fire grupper med 12-13 hold i hver gruppe. Det betyder at I forhåbentlig når at se alle holdene fra en gruppe, før den næste dukker op.\\
For at holdet har ret til at løse jeres post, skal de være igang med jeres rute, og skal have løst de foregående ruter. I kan se om ruterne før jeres egen, har fået et hul på holdenes ID-kort.
\subsection{App og tjek ind}
På www.post25.sagalobet.dk skal I indberette holdenes point samt tjekke deltagerne ind og ud.\\
I skal logge ind med jeres postnummer - Altså \textbf{C2}\\
\newline
Kommer I til at lave en forkert registrering, skal I kontakte mål med det samme, på tlf. 3267 9499.
\newpage
\vspace*{.4cm}
\subsection{Postbeskrivelser}
I har fået udleveret 5 plader med postbeskrivelser til jeres post. På hver side af pladen er der en postbeskrivelse til enten væbner- eller seniorhold. I kan se på bjælken i toppen, om det er til væbner- eller seniorhold. Vi har i udgangspunktet ikke lavet andet end sproglige tilretninger på posten, men læs den lige igennem, så I er sikre på at vide det samme som deltagerne ved.\\
Vi vil \textbf{meget gerne} have pladerne retur i målområdet efter løbet.
\subsection{Sikkerhed på løbet}
For at give deltagerne en tryg oplevelse har vi nogle sikkerhedskrav til jeres post:
\begin{itemize}
  \item I skal opsætte jeres velkomstskilt, så det er det første deltagerne ser, når de kommer tæt på posten.
  \item I skal alle sammen bære de udleverede ID-kort yderst, så de er tydelige for deltagerne.
  \item I skal udlevere postbeskrivelserne til deltagerne, så de ved, at I er en ægte post.
  \item I skal huske at tjekke deltagerne ind og ud så snart de ankommer og forlader posten, så vi kan følge med i, hvor langt de er.
\end{itemize}
\subsection{Point}
Alle levende poster giver mellem 0 og 100 point, med mulighed for at optjene 25 bonuspoint, hvis posten løses på en bestemt måde. Giv kun deltagerne point ud fra deres opgaveløsning og ikke ud fra om de fedter for jer. Når et hold har gennemført posten, skal pointene indberettes i appen. Husk, at I ikke kan lave om i jeres registreringer, og det kun er muligt at indsende point én gang pr. hold.\\
\newline
Kommer I til at lave en forkert registrering, skal I kontakte mål med det samme, på tlf. 3267 9499.
\subsection{Åbningstider}
Alle poster åbner kl. 16:00 og lukker kl. 08:00.\\
Tjekker et hold ind på jeres post inden kl. 08.00, må de stadig gerne løse posten.\\
Hold, som ankommer efter kl. 08:00, kan ikke løse posten.\\
Basen lukker også kl. 08:00.
\subsection{Postforløb}
For at give deltagerne den samme oplevelse, vil vi gerne have jer til at følge dette postforløb:
\begin{itemize}
  \item Byd holdet velkommen til posten.
  \item Modtag holdets ID-kort og tjek dem ind på app\'en.
  \item Udlever postbeskrivelsen til holdet, hvis der ikke er kø, eller de er forrest i køen. Hvis holdet skal sidde i kø, skal I fortælle dem, hvor lang tid de kan forvente at vente.
  \item Sæt holdet i gang med at løse posten, når de har læst postbeskrivelsen.
  \item Fortæl holdet, hvor mange point de har tjent, når de er færdige med at løse posten.
  \item Registrer holdets point på app\'en.
  \item Tjek holdet ud på app\'en og giv dem deres ID-kort tilbage.
  \item Hold øje med om holdet er ved at gå kold. Giv dem noget opbakning, hvis det er tilfældet.
\end{itemize}
\newpage
\vspace*{.4cm}
\subsection{Baseområdet}
Hvis der er brug for at sende noget af jeres mandskab ind på en pause, kan de finde sig til rette i postområdet på basen. Husk at holdene kan komme hele natten, så sørg for ikke at sende hele postens mandskab afsted på en gang.\\
På basen er der en del af området, som er forbeholdt udvalget. Vi vil meget gerne hyggesnakke med jer og svare på spørgsmål, men vi har brug for ro i udvalgsområdet. Det håber, vi at I vil respektere. Har I brug for at nogen fra udvalget kommer ud, kan I sige det til de udvalgsmedlemmer, som bemander basebordet.
\subsection{Hvis en deltager vil udgå af løbet}
Undervejs på løbet kommer I til at opleve mange trætte børn og unge. Nogle af dem vil måske også gerne give op. Hvis det er tilfældet, skal I først spørge om de er sikre. Fortæl deltagerne, hvor langt der ca. er til næste post og hvem der bemander den. Det kan være, at det er deres egen kreds og at de gerne vil nå forbi den post også. Hvis deltageren stadig ønsker at give op, skal I finde et sted, de kan holde sig i læ og så ringe til mål på tlf. 3267 9499. Efterfølgende henter vi deltageren, når det er muligt. Deltagere som ønsker at give op, må gerne hjælpe sit hold med at løse posten.\\
\newline
Oplever I deltagere, hvor I selv tænker, at det vil være bedst for deltageren at udgå at løbet, så kontakt udvalget. Det er kun deltagerne eller udvalget der kan træffe beslutning om at udgå af løbet.
\subsection{Følg med i løbet}
I app\'en kan I se, hvor langt holdene er nået og se om der er nogen på vej til jeres post.
\subsection{Postpokalen}
Alle løbets levende poster bliver vurderet af deltagerne. Posterne får point på en skala fra 1-10.\\
Den samlede score udregnes som et gennemsnit af de givne point. For at vi får den bedst mulige konkurrence, må I meget gerne minde deltagerne om at give jer point til postpokalen, inden de tjekker ud.\\
Undgå også at fedte for deltagerne med slik, snacks eller andet. - I giver heller ikke deltagerne fedtepoint for at fedte for jer.\\
\newline
Udover deltagerne, vil forskellige jury-medlemmer dukke op i løbet af løbet for at vurdere jeres post. Vi håber at I vil tage godt imod dem.
\subsection{Hemmeligheder}
En stor del af de point, deltagerne kan opnå på Sagaløbet, findes som hemmelige og skjulte point. Opdager I sådan nogle, så lad være med at fortælle deltagerne om dem - også de deltagere fra jeres egen kreds. På den måde får vi den mest fair konkurrence.
\newpage
\vspace*{.4cm}
\subsection{Mærker}
Når vi er færdige med løbet, kan alt postmandskab, som ikke allerede har et mærke, få et SagaCrew-mærke.\\
Alle deltagere som gennemfører Sagaløbet får også et mærke. Deltagere som tidligere har gennemført løbet, men kun fået ét mærke, kan købe ekstra mærker efter præmieoverrækkelsen til 10 kr. stykket.
\subsection{Kontakt}
Får I brug for at komme i kontakt med udvalget gennem natten, så ring til mål på tlf. 3267 9499. Har I brug for at få fat på et bestemt udvalgsmedlem, kan vores kontaktoplysninger findes på hjemmesiden. Vi ser dog helst at I kontakter mål frem for at kontakte enkelte udvalgsmedlemmer, da vi ordner koordinationen herfra.
\newline
Til sidst er der ikke så meget andet at sige end tak!\begin{itemize}
  \item fordi I er med til at skabe Sagaløbet 2025
  \item for at være gode ambassadører for vores arrangement
  \item for at tage jeres væbnere, seniorvæbnere og seniorer med på løb
  \item for at sikre at Sagaløbet er en fair konkurrence
  \item for at give deltagerne uforglemmelige oplevelser, som giver dem mulighed for at vokse
  \item for at hjælpe os med at udvikle løbet
  \item for at give os muligheden for at skabe fantastiske FDF-oplevelser i fællesskab
  \item for alt det vi ikke får sagt tak for
\end{itemize}
Vi glæder os helt vildt til at besøge jer i løbet af natten, og til at opleve løbet sammen med jer og deltagerne!
\newline
Rigtig godt løb!\\
\newline
\textcolor{søblå}{De bedste hilsner}\\
\textcolor{natblå}{\textbf{Alberte, Augusta, Charlotte, Dan, Jens Peter, Jonas, Lasse, Lucas, Magnus \& Thor}}\\
\textcolor{natblå}{\textbf{Sagaløbsudvalget}}\\
\newpage
\title{Postinfo}\\
Kære \textbf{Kærby \& Visse}\\
\newline
Velkommen på Sagaløbet 2025.\\
Vi har glædet os vildt til at lave løb med jer.\\
\newline
I har fået \textbf{Post C3}. Posten er placeret på \textbf{RUTE C}\\
Vi har, som mange tidligere år, forsøgt at gøre løbet bedre for både deltagere og jer. I år drejer det sig om småjusteringer, så hvis I har været med før, så er det meste, som I kender det.\\
\newline
Hvis I ikke har været med før, så håber vi, at dette infobrev kan hjælpe jer med at være klædt godt på. Har I spørgsmål efter I har læst brevet, så sig til! Ring til mål på tlf. +45 3267 9499.
\subsection{Sagaløbets principper}
Til planlægningen af årets løb har vi skabt seks principper, som er det bærende element for både planlægning, afviklingen og udviklingen af Sagaløbet.\\
De seks principper er:
\begin{itemize}
  \item \textbf{Vi skaber Sagaløbet sammen}
  \item \textbf{Alle skal opleve at lykkes}
  \item \textbf{Alle skal opleve udfordringer}
  \item \textbf{Det kendte og ukendte}
  \item \textbf{Vi gør os umage}
  \item \textbf{Retfærdig konkurrence}
\end{itemize}
Vi håber, at I kan genkende at Sagaløbet er bygget op om disse seks bærende principper. Det bliver også med disse seks principper, at vi kommer til at træffe beslutninger under løbet. Har I spørgsmål til principperne, er vi altid klar til at tage en snak om dem.
\subsection{Ruter og deltagerstrøm}
Løbet er inddelt i fire mindre ruter. Hver rute er angivet med et bogstav og har 4 levende poster og én død post. Derudover er der 4 levende seniorposter i målområdet.\\
Ved at inddele løbet i fire mindre løb kan vi sprede deltagerne ud, have kontakt med dem løbende, give dem et helle og sikre et bedre flow i løbet.\\
\newline
Det betyder at I skal stå post fra 16:00 til 08:00.\\
Det er lang tid, og det kan være nødvendigt at hjælpe hinanden med at holde pauser undervejs.\\
\newline
Holdene er delt op i fire grupper med 12-13 hold i hver gruppe. Det betyder at I forhåbentlig når at se alle holdene fra en gruppe, før den næste dukker op.\\
For at holdet har ret til at løse jeres post, skal de være igang med jeres rute, og skal have løst de foregående ruter. I kan se om ruterne før jeres egen, har fået et hul på holdenes ID-kort.
\subsection{App og tjek ind}
På www.post25.sagalobet.dk skal I indberette holdenes point samt tjekke deltagerne ind og ud.\\
I skal logge ind med jeres postnummer - Altså \textbf{C3}\\
\newline
Kommer I til at lave en forkert registrering, skal I kontakte mål med det samme, på tlf. 3267 9499.
\newpage
\vspace*{.4cm}
\subsection{Postbeskrivelser}
I har fået udleveret 5 plader med postbeskrivelser til jeres post. På hver side af pladen er der en postbeskrivelse til enten væbner- eller seniorhold. I kan se på bjælken i toppen, om det er til væbner- eller seniorhold. Vi har i udgangspunktet ikke lavet andet end sproglige tilretninger på posten, men læs den lige igennem, så I er sikre på at vide det samme som deltagerne ved.\\
Vi vil \textbf{meget gerne} have pladerne retur i målområdet efter løbet.
\subsection{Sikkerhed på løbet}
For at give deltagerne en tryg oplevelse har vi nogle sikkerhedskrav til jeres post:
\begin{itemize}
  \item I skal opsætte jeres velkomstskilt, så det er det første deltagerne ser, når de kommer tæt på posten.
  \item I skal alle sammen bære de udleverede ID-kort yderst, så de er tydelige for deltagerne.
  \item I skal udlevere postbeskrivelserne til deltagerne, så de ved, at I er en ægte post.
  \item I skal huske at tjekke deltagerne ind og ud så snart de ankommer og forlader posten, så vi kan følge med i, hvor langt de er.
\end{itemize}
\subsection{Point}
Alle levende poster giver mellem 0 og 100 point, med mulighed for at optjene 25 bonuspoint, hvis posten løses på en bestemt måde. Giv kun deltagerne point ud fra deres opgaveløsning og ikke ud fra om de fedter for jer. Når et hold har gennemført posten, skal pointene indberettes i appen. Husk, at I ikke kan lave om i jeres registreringer, og det kun er muligt at indsende point én gang pr. hold.\\
\newline
Kommer I til at lave en forkert registrering, skal I kontakte mål med det samme, på tlf. 3267 9499.
\subsection{Åbningstider}
Alle poster åbner kl. 16:00 og lukker kl. 08:00.\\
Tjekker et hold ind på jeres post inden kl. 08.00, må de stadig gerne løse posten.\\
Hold, som ankommer efter kl. 08:00, kan ikke løse posten.\\
Basen lukker også kl. 08:00.
\subsection{Postforløb}
For at give deltagerne den samme oplevelse, vil vi gerne have jer til at følge dette postforløb:
\begin{itemize}
  \item Byd holdet velkommen til posten.
  \item Modtag holdets ID-kort og tjek dem ind på app\'en.
  \item Udlever postbeskrivelsen til holdet, hvis der ikke er kø, eller de er forrest i køen. Hvis holdet skal sidde i kø, skal I fortælle dem, hvor lang tid de kan forvente at vente.
  \item Sæt holdet i gang med at løse posten, når de har læst postbeskrivelsen.
  \item Fortæl holdet, hvor mange point de har tjent, når de er færdige med at løse posten.
  \item Registrer holdets point på app\'en.
  \item Tjek holdet ud på app\'en og giv dem deres ID-kort tilbage.
  \item Hold øje med om holdet er ved at gå kold. Giv dem noget opbakning, hvis det er tilfældet.
\end{itemize}
\newpage
\vspace*{.4cm}
\subsection{Baseområdet}
Hvis der er brug for at sende noget af jeres mandskab ind på en pause, kan de finde sig til rette i postområdet på basen. Husk at holdene kan komme hele natten, så sørg for ikke at sende hele postens mandskab afsted på en gang.\\
På basen er der en del af området, som er forbeholdt udvalget. Vi vil meget gerne hyggesnakke med jer og svare på spørgsmål, men vi har brug for ro i udvalgsområdet. Det håber, vi at I vil respektere. Har I brug for at nogen fra udvalget kommer ud, kan I sige det til de udvalgsmedlemmer, som bemander basebordet.
\subsection{Hvis en deltager vil udgå af løbet}
Undervejs på løbet kommer I til at opleve mange trætte børn og unge. Nogle af dem vil måske også gerne give op. Hvis det er tilfældet, skal I først spørge om de er sikre. Fortæl deltagerne, hvor langt der ca. er til næste post og hvem der bemander den. Det kan være, at det er deres egen kreds og at de gerne vil nå forbi den post også. Hvis deltageren stadig ønsker at give op, skal I finde et sted, de kan holde sig i læ og så ringe til mål på tlf. 3267 9499. Efterfølgende henter vi deltageren, når det er muligt. Deltagere som ønsker at give op, må gerne hjælpe sit hold med at løse posten.\\
\newline
Oplever I deltagere, hvor I selv tænker, at det vil være bedst for deltageren at udgå at løbet, så kontakt udvalget. Det er kun deltagerne eller udvalget der kan træffe beslutning om at udgå af løbet.
\subsection{Følg med i løbet}
I app\'en kan I se, hvor langt holdene er nået og se om der er nogen på vej til jeres post.
\subsection{Postpokalen}
Alle løbets levende poster bliver vurderet af deltagerne. Posterne får point på en skala fra 1-10.\\
Den samlede score udregnes som et gennemsnit af de givne point. For at vi får den bedst mulige konkurrence, må I meget gerne minde deltagerne om at give jer point til postpokalen, inden de tjekker ud.\\
Undgå også at fedte for deltagerne med slik, snacks eller andet. - I giver heller ikke deltagerne fedtepoint for at fedte for jer.\\
\newline
Udover deltagerne, vil forskellige jury-medlemmer dukke op i løbet af løbet for at vurdere jeres post. Vi håber at I vil tage godt imod dem.
\subsection{Hemmeligheder}
En stor del af de point, deltagerne kan opnå på Sagaløbet, findes som hemmelige og skjulte point. Opdager I sådan nogle, så lad være med at fortælle deltagerne om dem - også de deltagere fra jeres egen kreds. På den måde får vi den mest fair konkurrence.
\newpage
\vspace*{.4cm}
\subsection{Mærker}
Når vi er færdige med løbet, kan alt postmandskab, som ikke allerede har et mærke, få et SagaCrew-mærke.\\
Alle deltagere som gennemfører Sagaløbet får også et mærke. Deltagere som tidligere har gennemført løbet, men kun fået ét mærke, kan købe ekstra mærker efter præmieoverrækkelsen til 10 kr. stykket.
\subsection{Kontakt}
Får I brug for at komme i kontakt med udvalget gennem natten, så ring til mål på tlf. 3267 9499. Har I brug for at få fat på et bestemt udvalgsmedlem, kan vores kontaktoplysninger findes på hjemmesiden. Vi ser dog helst at I kontakter mål frem for at kontakte enkelte udvalgsmedlemmer, da vi ordner koordinationen herfra.
\newline
Til sidst er der ikke så meget andet at sige end tak!\begin{itemize}
  \item fordi I er med til at skabe Sagaløbet 2025
  \item for at være gode ambassadører for vores arrangement
  \item for at tage jeres væbnere, seniorvæbnere og seniorer med på løb
  \item for at sikre at Sagaløbet er en fair konkurrence
  \item for at give deltagerne uforglemmelige oplevelser, som giver dem mulighed for at vokse
  \item for at hjælpe os med at udvikle løbet
  \item for at give os muligheden for at skabe fantastiske FDF-oplevelser i fællesskab
  \item for alt det vi ikke får sagt tak for
\end{itemize}
Vi glæder os helt vildt til at besøge jer i løbet af natten, og til at opleve løbet sammen med jer og deltagerne!
\newline
Rigtig godt løb!\\
\newline
\textcolor{søblå}{De bedste hilsner}\\
\textcolor{natblå}{\textbf{Alberte, Augusta, Charlotte, Dan, Jens Peter, Jonas, Lasse, Lucas, Magnus \& Thor}}\\
\textcolor{natblå}{\textbf{Sagaløbsudvalget}}\\
\newpage
\title{Postinfo}\\
Kære \textbf{Frederikshavn 1}\\
\newline
Velkommen på Sagaløbet 2025.\\
Vi har glædet os vildt til at lave løb med jer.\\
\newline
I har fået \textbf{Post C4}. Posten er placeret på \textbf{RUTE C}\\
Vi har, som mange tidligere år, forsøgt at gøre løbet bedre for både deltagere og jer. I år drejer det sig om småjusteringer, så hvis I har været med før, så er det meste, som I kender det.\\
\newline
Hvis I ikke har været med før, så håber vi, at dette infobrev kan hjælpe jer med at være klædt godt på. Har I spørgsmål efter I har læst brevet, så sig til! Ring til mål på tlf. +45 3267 9499.
\subsection{Sagaløbets principper}
Til planlægningen af årets løb har vi skabt seks principper, som er det bærende element for både planlægning, afviklingen og udviklingen af Sagaløbet.\\
De seks principper er:
\begin{itemize}
  \item \textbf{Vi skaber Sagaløbet sammen}
  \item \textbf{Alle skal opleve at lykkes}
  \item \textbf{Alle skal opleve udfordringer}
  \item \textbf{Det kendte og ukendte}
  \item \textbf{Vi gør os umage}
  \item \textbf{Retfærdig konkurrence}
\end{itemize}
Vi håber, at I kan genkende at Sagaløbet er bygget op om disse seks bærende principper. Det bliver også med disse seks principper, at vi kommer til at træffe beslutninger under løbet. Har I spørgsmål til principperne, er vi altid klar til at tage en snak om dem.
\subsection{Ruter og deltagerstrøm}
Løbet er inddelt i fire mindre ruter. Hver rute er angivet med et bogstav og har 4 levende poster og én død post. Derudover er der 4 levende seniorposter i målområdet.\\
Ved at inddele løbet i fire mindre løb kan vi sprede deltagerne ud, have kontakt med dem løbende, give dem et helle og sikre et bedre flow i løbet.\\
\newline
Det betyder at I skal stå post fra 16:00 til 08:00.\\
Det er lang tid, og det kan være nødvendigt at hjælpe hinanden med at holde pauser undervejs.\\
\newline
Holdene er delt op i fire grupper med 12-13 hold i hver gruppe. Det betyder at I forhåbentlig når at se alle holdene fra en gruppe, før den næste dukker op.\\
For at holdet har ret til at løse jeres post, skal de være igang med jeres rute, og skal have løst de foregående ruter. I kan se om ruterne før jeres egen, har fået et hul på holdenes ID-kort.
\subsection{App og tjek ind}
På www.post25.sagalobet.dk skal I indberette holdenes point samt tjekke deltagerne ind og ud.\\
I skal logge ind med jeres postnummer - Altså \textbf{C4}\\
\newline
Kommer I til at lave en forkert registrering, skal I kontakte mål med det samme, på tlf. 3267 9499.
\newpage
\vspace*{.4cm}
\subsection{Postbeskrivelser}
I har fået udleveret 5 plader med postbeskrivelser til jeres post. På hver side af pladen er der en postbeskrivelse til enten væbner- eller seniorhold. I kan se på bjælken i toppen, om det er til væbner- eller seniorhold. Vi har i udgangspunktet ikke lavet andet end sproglige tilretninger på posten, men læs den lige igennem, så I er sikre på at vide det samme som deltagerne ved.\\
Vi vil \textbf{meget gerne} have pladerne retur i målområdet efter løbet.
\subsection{Sikkerhed på løbet}
For at give deltagerne en tryg oplevelse har vi nogle sikkerhedskrav til jeres post:
\begin{itemize}
  \item I skal opsætte jeres velkomstskilt, så det er det første deltagerne ser, når de kommer tæt på posten.
  \item I skal alle sammen bære de udleverede ID-kort yderst, så de er tydelige for deltagerne.
  \item I skal udlevere postbeskrivelserne til deltagerne, så de ved, at I er en ægte post.
  \item I skal huske at tjekke deltagerne ind og ud så snart de ankommer og forlader posten, så vi kan følge med i, hvor langt de er.
\end{itemize}
\subsection{Point}
Alle levende poster giver mellem 0 og 100 point, med mulighed for at optjene 25 bonuspoint, hvis posten løses på en bestemt måde. Giv kun deltagerne point ud fra deres opgaveløsning og ikke ud fra om de fedter for jer. Når et hold har gennemført posten, skal pointene indberettes i appen. Husk, at I ikke kan lave om i jeres registreringer, og det kun er muligt at indsende point én gang pr. hold.\\
\newline
Kommer I til at lave en forkert registrering, skal I kontakte mål med det samme, på tlf. 3267 9499.
\subsection{Åbningstider}
Alle poster åbner kl. 16:00 og lukker kl. 08:00.\\
Tjekker et hold ind på jeres post inden kl. 08.00, må de stadig gerne løse posten.\\
Hold, som ankommer efter kl. 08:00, kan ikke løse posten.\\
Basen lukker også kl. 08:00.
\subsection{Postforløb}
For at give deltagerne den samme oplevelse, vil vi gerne have jer til at følge dette postforløb:
\begin{itemize}
  \item Byd holdet velkommen til posten.
  \item Modtag holdets ID-kort og tjek dem ind på app\'en.
  \item Udlever postbeskrivelsen til holdet, hvis der ikke er kø, eller de er forrest i køen. Hvis holdet skal sidde i kø, skal I fortælle dem, hvor lang tid de kan forvente at vente.
  \item Sæt holdet i gang med at løse posten, når de har læst postbeskrivelsen.
  \item Fortæl holdet, hvor mange point de har tjent, når de er færdige med at løse posten.
  \item Registrer holdets point på app\'en.
  \item Tjek holdet ud på app\'en og giv dem deres ID-kort tilbage.
  \item Hold øje med om holdet er ved at gå kold. Giv dem noget opbakning, hvis det er tilfældet.
\end{itemize}
\newpage
\vspace*{.4cm}
\subsection{Baseområdet}
Hvis der er brug for at sende noget af jeres mandskab ind på en pause, kan de finde sig til rette i postområdet på basen. Husk at holdene kan komme hele natten, så sørg for ikke at sende hele postens mandskab afsted på en gang.\\
På basen er der en del af området, som er forbeholdt udvalget. Vi vil meget gerne hyggesnakke med jer og svare på spørgsmål, men vi har brug for ro i udvalgsområdet. Det håber, vi at I vil respektere. Har I brug for at nogen fra udvalget kommer ud, kan I sige det til de udvalgsmedlemmer, som bemander basebordet.
\subsection{Hvis en deltager vil udgå af løbet}
Undervejs på løbet kommer I til at opleve mange trætte børn og unge. Nogle af dem vil måske også gerne give op. Hvis det er tilfældet, skal I først spørge om de er sikre. Fortæl deltagerne, hvor langt der ca. er til næste post og hvem der bemander den. Det kan være, at det er deres egen kreds og at de gerne vil nå forbi den post også. Hvis deltageren stadig ønsker at give op, skal I finde et sted, de kan holde sig i læ og så ringe til mål på tlf. 3267 9499. Efterfølgende henter vi deltageren, når det er muligt. Deltagere som ønsker at give op, må gerne hjælpe sit hold med at løse posten.\\
\newline
Oplever I deltagere, hvor I selv tænker, at det vil være bedst for deltageren at udgå at løbet, så kontakt udvalget. Det er kun deltagerne eller udvalget der kan træffe beslutning om at udgå af løbet.
\subsection{Følg med i løbet}
I app\'en kan I se, hvor langt holdene er nået og se om der er nogen på vej til jeres post.
\subsection{Postpokalen}
Alle løbets levende poster bliver vurderet af deltagerne. Posterne får point på en skala fra 1-10.\\
Den samlede score udregnes som et gennemsnit af de givne point. For at vi får den bedst mulige konkurrence, må I meget gerne minde deltagerne om at give jer point til postpokalen, inden de tjekker ud.\\
Undgå også at fedte for deltagerne med slik, snacks eller andet. - I giver heller ikke deltagerne fedtepoint for at fedte for jer.\\
\newline
Udover deltagerne, vil forskellige jury-medlemmer dukke op i løbet af løbet for at vurdere jeres post. Vi håber at I vil tage godt imod dem.
\subsection{Hemmeligheder}
En stor del af de point, deltagerne kan opnå på Sagaløbet, findes som hemmelige og skjulte point. Opdager I sådan nogle, så lad være med at fortælle deltagerne om dem - også de deltagere fra jeres egen kreds. På den måde får vi den mest fair konkurrence.
\newpage
\vspace*{.4cm}
\subsection{Mærker}
Når vi er færdige med løbet, kan alt postmandskab, som ikke allerede har et mærke, få et SagaCrew-mærke.\\
Alle deltagere som gennemfører Sagaløbet får også et mærke. Deltagere som tidligere har gennemført løbet, men kun fået ét mærke, kan købe ekstra mærker efter præmieoverrækkelsen til 10 kr. stykket.
\subsection{Kontakt}
Får I brug for at komme i kontakt med udvalget gennem natten, så ring til mål på tlf. 3267 9499. Har I brug for at få fat på et bestemt udvalgsmedlem, kan vores kontaktoplysninger findes på hjemmesiden. Vi ser dog helst at I kontakter mål frem for at kontakte enkelte udvalgsmedlemmer, da vi ordner koordinationen herfra.
\newline
Til sidst er der ikke så meget andet at sige end tak!\begin{itemize}
  \item fordi I er med til at skabe Sagaløbet 2025
  \item for at være gode ambassadører for vores arrangement
  \item for at tage jeres væbnere, seniorvæbnere og seniorer med på løb
  \item for at sikre at Sagaløbet er en fair konkurrence
  \item for at give deltagerne uforglemmelige oplevelser, som giver dem mulighed for at vokse
  \item for at hjælpe os med at udvikle løbet
  \item for at give os muligheden for at skabe fantastiske FDF-oplevelser i fællesskab
  \item for alt det vi ikke får sagt tak for
\end{itemize}
Vi glæder os helt vildt til at besøge jer i løbet af natten, og til at opleve løbet sammen med jer og deltagerne!
\newline
Rigtig godt løb!\\
\newline
\textcolor{søblå}{De bedste hilsner}\\
\textcolor{natblå}{\textbf{Alberte, Augusta, Charlotte, Dan, Jens Peter, Jonas, Lasse, Lucas, Magnus \& Thor}}\\
\textcolor{natblå}{\textbf{Sagaløbsudvalget}}\\
\newpage
\title{Postinfo}\\
Kære \textbf{Gug}\\
\newline
Velkommen på Sagaløbet 2025.\\
Vi har glædet os vildt til at lave løb med jer.\\
\newline
I har fået \textbf{Post D1}. Posten er placeret på \textbf{RUTE D}\\
Vi har, som mange tidligere år, forsøgt at gøre løbet bedre for både deltagere og jer. I år drejer det sig om småjusteringer, så hvis I har været med før, så er det meste, som I kender det.\\
\newline
Hvis I ikke har været med før, så håber vi, at dette infobrev kan hjælpe jer med at være klædt godt på. Har I spørgsmål efter I har læst brevet, så sig til! Ring til mål på tlf. +45 3267 9499.
\subsection{Sagaløbets principper}
Til planlægningen af årets løb har vi skabt seks principper, som er det bærende element for både planlægning, afviklingen og udviklingen af Sagaløbet.\\
De seks principper er:
\begin{itemize}
  \item \textbf{Vi skaber Sagaløbet sammen}
  \item \textbf{Alle skal opleve at lykkes}
  \item \textbf{Alle skal opleve udfordringer}
  \item \textbf{Det kendte og ukendte}
  \item \textbf{Vi gør os umage}
  \item \textbf{Retfærdig konkurrence}
\end{itemize}
Vi håber, at I kan genkende at Sagaløbet er bygget op om disse seks bærende principper. Det bliver også med disse seks principper, at vi kommer til at træffe beslutninger under løbet. Har I spørgsmål til principperne, er vi altid klar til at tage en snak om dem.
\subsection{Ruter og deltagerstrøm}
Løbet er inddelt i fire mindre ruter. Hver rute er angivet med et bogstav og har 4 levende poster og én død post. Derudover er der 4 levende seniorposter i målområdet.\\
Ved at inddele løbet i fire mindre løb kan vi sprede deltagerne ud, have kontakt med dem løbende, give dem et helle og sikre et bedre flow i løbet.\\
\newline
Det betyder at I skal stå post fra 16:00 til 08:00.\\
Det er lang tid, og det kan være nødvendigt at hjælpe hinanden med at holde pauser undervejs.\\
\newline
Holdene er delt op i fire grupper med 12-13 hold i hver gruppe. Det betyder at I forhåbentlig når at se alle holdene fra en gruppe, før den næste dukker op.\\
For at holdet har ret til at løse jeres post, skal de være igang med jeres rute, og skal have løst de foregående ruter. I kan se om ruterne før jeres egen, har fået et hul på holdenes ID-kort.
\subsection{App og tjek ind}
På www.post25.sagalobet.dk skal I indberette holdenes point samt tjekke deltagerne ind og ud.\\
I skal logge ind med jeres postnummer - Altså \textbf{D1}\\
\newline
Kommer I til at lave en forkert registrering, skal I kontakte mål med det samme, på tlf. 3267 9499.
\newpage
\vspace*{.4cm}
\subsection{Postbeskrivelser}
I har fået udleveret 5 plader med postbeskrivelser til jeres post. På hver side af pladen er der en postbeskrivelse til enten væbner- eller seniorhold. I kan se på bjælken i toppen, om det er til væbner- eller seniorhold. Vi har i udgangspunktet ikke lavet andet end sproglige tilretninger på posten, men læs den lige igennem, så I er sikre på at vide det samme som deltagerne ved.\\
Vi vil \textbf{meget gerne} have pladerne retur i målområdet efter løbet.
\subsection{Sikkerhed på løbet}
For at give deltagerne en tryg oplevelse har vi nogle sikkerhedskrav til jeres post:
\begin{itemize}
  \item I skal opsætte jeres velkomstskilt, så det er det første deltagerne ser, når de kommer tæt på posten.
  \item I skal alle sammen bære de udleverede ID-kort yderst, så de er tydelige for deltagerne.
  \item I skal udlevere postbeskrivelserne til deltagerne, så de ved, at I er en ægte post.
  \item I skal huske at tjekke deltagerne ind og ud så snart de ankommer og forlader posten, så vi kan følge med i, hvor langt de er.
\end{itemize}
\subsection{Point}
Alle levende poster giver mellem 0 og 100 point, med mulighed for at optjene 25 bonuspoint, hvis posten løses på en bestemt måde. Giv kun deltagerne point ud fra deres opgaveløsning og ikke ud fra om de fedter for jer. Når et hold har gennemført posten, skal pointene indberettes i appen. Husk, at I ikke kan lave om i jeres registreringer, og det kun er muligt at indsende point én gang pr. hold.\\
\newline
Kommer I til at lave en forkert registrering, skal I kontakte mål med det samme, på tlf. 3267 9499.
\subsection{Åbningstider}
Alle poster åbner kl. 16:00 og lukker kl. 08:00.\\
Tjekker et hold ind på jeres post inden kl. 08.00, må de stadig gerne løse posten.\\
Hold, som ankommer efter kl. 08:00, kan ikke løse posten.\\
Basen lukker også kl. 08:00.
\subsection{Postforløb}
For at give deltagerne den samme oplevelse, vil vi gerne have jer til at følge dette postforløb:
\begin{itemize}
  \item Byd holdet velkommen til posten.
  \item Modtag holdets ID-kort og tjek dem ind på app\'en.
  \item Udlever postbeskrivelsen til holdet, hvis der ikke er kø, eller de er forrest i køen. Hvis holdet skal sidde i kø, skal I fortælle dem, hvor lang tid de kan forvente at vente.
  \item Sæt holdet i gang med at løse posten, når de har læst postbeskrivelsen.
  \item Fortæl holdet, hvor mange point de har tjent, når de er færdige med at løse posten.
  \item Registrer holdets point på app\'en.
  \item Tjek holdet ud på app\'en og giv dem deres ID-kort tilbage.
  \item Hold øje med om holdet er ved at gå kold. Giv dem noget opbakning, hvis det er tilfældet.
\end{itemize}
\newpage
\vspace*{.4cm}
\subsection{Baseområdet}
Hvis der er brug for at sende noget af jeres mandskab ind på en pause, kan de finde sig til rette i postområdet på basen. Husk at holdene kan komme hele natten, så sørg for ikke at sende hele postens mandskab afsted på en gang.\\
På basen er der en del af området, som er forbeholdt udvalget. Vi vil meget gerne hyggesnakke med jer og svare på spørgsmål, men vi har brug for ro i udvalgsområdet. Det håber, vi at I vil respektere. Har I brug for at nogen fra udvalget kommer ud, kan I sige det til de udvalgsmedlemmer, som bemander basebordet.
\subsection{Hvis en deltager vil udgå af løbet}
Undervejs på løbet kommer I til at opleve mange trætte børn og unge. Nogle af dem vil måske også gerne give op. Hvis det er tilfældet, skal I først spørge om de er sikre. Fortæl deltagerne, hvor langt der ca. er til næste post og hvem der bemander den. Det kan være, at det er deres egen kreds og at de gerne vil nå forbi den post også. Hvis deltageren stadig ønsker at give op, skal I finde et sted, de kan holde sig i læ og så ringe til mål på tlf. 3267 9499. Efterfølgende henter vi deltageren, når det er muligt. Deltagere som ønsker at give op, må gerne hjælpe sit hold med at løse posten.\\
\newline
Oplever I deltagere, hvor I selv tænker, at det vil være bedst for deltageren at udgå at løbet, så kontakt udvalget. Det er kun deltagerne eller udvalget der kan træffe beslutning om at udgå af løbet.
\subsection{Følg med i løbet}
I app\'en kan I se, hvor langt holdene er nået og se om der er nogen på vej til jeres post.
\subsection{Postpokalen}
Alle løbets levende poster bliver vurderet af deltagerne. Posterne får point på en skala fra 1-10.\\
Den samlede score udregnes som et gennemsnit af de givne point. For at vi får den bedst mulige konkurrence, må I meget gerne minde deltagerne om at give jer point til postpokalen, inden de tjekker ud.\\
Undgå også at fedte for deltagerne med slik, snacks eller andet. - I giver heller ikke deltagerne fedtepoint for at fedte for jer.\\
\newline
Udover deltagerne, vil forskellige jury-medlemmer dukke op i løbet af løbet for at vurdere jeres post. Vi håber at I vil tage godt imod dem.
\subsection{Hemmeligheder}
En stor del af de point, deltagerne kan opnå på Sagaløbet, findes som hemmelige og skjulte point. Opdager I sådan nogle, så lad være med at fortælle deltagerne om dem - også de deltagere fra jeres egen kreds. På den måde får vi den mest fair konkurrence.
\newpage
\vspace*{.4cm}
\subsection{Mærker}
Når vi er færdige med løbet, kan alt postmandskab, som ikke allerede har et mærke, få et SagaCrew-mærke.\\
Alle deltagere som gennemfører Sagaløbet får også et mærke. Deltagere som tidligere har gennemført løbet, men kun fået ét mærke, kan købe ekstra mærker efter præmieoverrækkelsen til 10 kr. stykket.
\subsection{Kontakt}
Får I brug for at komme i kontakt med udvalget gennem natten, så ring til mål på tlf. 3267 9499. Har I brug for at få fat på et bestemt udvalgsmedlem, kan vores kontaktoplysninger findes på hjemmesiden. Vi ser dog helst at I kontakter mål frem for at kontakte enkelte udvalgsmedlemmer, da vi ordner koordinationen herfra.
\newline
Til sidst er der ikke så meget andet at sige end tak!\begin{itemize}
  \item fordi I er med til at skabe Sagaløbet 2025
  \item for at være gode ambassadører for vores arrangement
  \item for at tage jeres væbnere, seniorvæbnere og seniorer med på løb
  \item for at sikre at Sagaløbet er en fair konkurrence
  \item for at give deltagerne uforglemmelige oplevelser, som giver dem mulighed for at vokse
  \item for at hjælpe os med at udvikle løbet
  \item for at give os muligheden for at skabe fantastiske FDF-oplevelser i fællesskab
  \item for alt det vi ikke får sagt tak for
\end{itemize}
Vi glæder os helt vildt til at besøge jer i løbet af natten, og til at opleve løbet sammen med jer og deltagerne!
\newline
Rigtig godt løb!\\
\newline
\textcolor{søblå}{De bedste hilsner}\\
\textcolor{natblå}{\textbf{Alberte, Augusta, Charlotte, Dan, Jens Peter, Jonas, Lasse, Lucas, Magnus \& Thor}}\\
\textcolor{natblå}{\textbf{Sagaløbsudvalget}}\\
\newpage
\title{Postinfo}\\
Kære \textbf{Udbud}\\
\newline
Velkommen på Sagaløbet 2025.\\
Vi har glædet os vildt til at lave løb med jer.\\
\newline
I har fået \textbf{Post D2}. Posten er placeret på \textbf{RUTE D}\\
Vi har, som mange tidligere år, forsøgt at gøre løbet bedre for både deltagere og jer. I år drejer det sig om småjusteringer, så hvis I har været med før, så er det meste, som I kender det.\\
\newline
Hvis I ikke har været med før, så håber vi, at dette infobrev kan hjælpe jer med at være klædt godt på. Har I spørgsmål efter I har læst brevet, så sig til! Ring til mål på tlf. +45 3267 9499.
\subsection{Sagaløbets principper}
Til planlægningen af årets løb har vi skabt seks principper, som er det bærende element for både planlægning, afviklingen og udviklingen af Sagaløbet.\\
De seks principper er:
\begin{itemize}
  \item \textbf{Vi skaber Sagaløbet sammen}
  \item \textbf{Alle skal opleve at lykkes}
  \item \textbf{Alle skal opleve udfordringer}
  \item \textbf{Det kendte og ukendte}
  \item \textbf{Vi gør os umage}
  \item \textbf{Retfærdig konkurrence}
\end{itemize}
Vi håber, at I kan genkende at Sagaløbet er bygget op om disse seks bærende principper. Det bliver også med disse seks principper, at vi kommer til at træffe beslutninger under løbet. Har I spørgsmål til principperne, er vi altid klar til at tage en snak om dem.
\subsection{Ruter og deltagerstrøm}
Løbet er inddelt i fire mindre ruter. Hver rute er angivet med et bogstav og har 4 levende poster og én død post. Derudover er der 4 levende seniorposter i målområdet.\\
Ved at inddele løbet i fire mindre løb kan vi sprede deltagerne ud, have kontakt med dem løbende, give dem et helle og sikre et bedre flow i løbet.\\
\newline
Det betyder at I skal stå post fra 16:00 til 08:00.\\
Det er lang tid, og det kan være nødvendigt at hjælpe hinanden med at holde pauser undervejs.\\
\newline
Holdene er delt op i fire grupper med 12-13 hold i hver gruppe. Det betyder at I forhåbentlig når at se alle holdene fra en gruppe, før den næste dukker op.\\
For at holdet har ret til at løse jeres post, skal de være igang med jeres rute, og skal have løst de foregående ruter. I kan se om ruterne før jeres egen, har fået et hul på holdenes ID-kort.
\subsection{App og tjek ind}
På www.post25.sagalobet.dk skal I indberette holdenes point samt tjekke deltagerne ind og ud.\\
I skal logge ind med jeres postnummer - Altså \textbf{D2}\\
\newline
Kommer I til at lave en forkert registrering, skal I kontakte mål med det samme, på tlf. 3267 9499.
\newpage
\vspace*{.4cm}
\subsection{Postbeskrivelser}
I har fået udleveret 5 plader med postbeskrivelser til jeres post. På hver side af pladen er der en postbeskrivelse til enten væbner- eller seniorhold. I kan se på bjælken i toppen, om det er til væbner- eller seniorhold. Vi har i udgangspunktet ikke lavet andet end sproglige tilretninger på posten, men læs den lige igennem, så I er sikre på at vide det samme som deltagerne ved.\\
Vi vil \textbf{meget gerne} have pladerne retur i målområdet efter løbet.
\subsection{Sikkerhed på løbet}
For at give deltagerne en tryg oplevelse har vi nogle sikkerhedskrav til jeres post:
\begin{itemize}
  \item I skal opsætte jeres velkomstskilt, så det er det første deltagerne ser, når de kommer tæt på posten.
  \item I skal alle sammen bære de udleverede ID-kort yderst, så de er tydelige for deltagerne.
  \item I skal udlevere postbeskrivelserne til deltagerne, så de ved, at I er en ægte post.
  \item I skal huske at tjekke deltagerne ind og ud så snart de ankommer og forlader posten, så vi kan følge med i, hvor langt de er.
\end{itemize}
\subsection{Point}
Alle levende poster giver mellem 0 og 100 point, med mulighed for at optjene 25 bonuspoint, hvis posten løses på en bestemt måde. Giv kun deltagerne point ud fra deres opgaveløsning og ikke ud fra om de fedter for jer. Når et hold har gennemført posten, skal pointene indberettes i appen. Husk, at I ikke kan lave om i jeres registreringer, og det kun er muligt at indsende point én gang pr. hold.\\
\newline
Kommer I til at lave en forkert registrering, skal I kontakte mål med det samme, på tlf. 3267 9499.
\subsection{Åbningstider}
Alle poster åbner kl. 16:00 og lukker kl. 08:00.\\
Tjekker et hold ind på jeres post inden kl. 08.00, må de stadig gerne løse posten.\\
Hold, som ankommer efter kl. 08:00, kan ikke løse posten.\\
Basen lukker også kl. 08:00.
\subsection{Postforløb}
For at give deltagerne den samme oplevelse, vil vi gerne have jer til at følge dette postforløb:
\begin{itemize}
  \item Byd holdet velkommen til posten.
  \item Modtag holdets ID-kort og tjek dem ind på app\'en.
  \item Udlever postbeskrivelsen til holdet, hvis der ikke er kø, eller de er forrest i køen. Hvis holdet skal sidde i kø, skal I fortælle dem, hvor lang tid de kan forvente at vente.
  \item Sæt holdet i gang med at løse posten, når de har læst postbeskrivelsen.
  \item Fortæl holdet, hvor mange point de har tjent, når de er færdige med at løse posten.
  \item Registrer holdets point på app\'en.
  \item Tjek holdet ud på app\'en og giv dem deres ID-kort tilbage.
  \item Hold øje med om holdet er ved at gå kold. Giv dem noget opbakning, hvis det er tilfældet.
\end{itemize}
\newpage
\vspace*{.4cm}
\subsection{Baseområdet}
Hvis der er brug for at sende noget af jeres mandskab ind på en pause, kan de finde sig til rette i postområdet på basen. Husk at holdene kan komme hele natten, så sørg for ikke at sende hele postens mandskab afsted på en gang.\\
På basen er der en del af området, som er forbeholdt udvalget. Vi vil meget gerne hyggesnakke med jer og svare på spørgsmål, men vi har brug for ro i udvalgsområdet. Det håber, vi at I vil respektere. Har I brug for at nogen fra udvalget kommer ud, kan I sige det til de udvalgsmedlemmer, som bemander basebordet.
\subsection{Hvis en deltager vil udgå af løbet}
Undervejs på løbet kommer I til at opleve mange trætte børn og unge. Nogle af dem vil måske også gerne give op. Hvis det er tilfældet, skal I først spørge om de er sikre. Fortæl deltagerne, hvor langt der ca. er til næste post og hvem der bemander den. Det kan være, at det er deres egen kreds og at de gerne vil nå forbi den post også. Hvis deltageren stadig ønsker at give op, skal I finde et sted, de kan holde sig i læ og så ringe til mål på tlf. 3267 9499. Efterfølgende henter vi deltageren, når det er muligt. Deltagere som ønsker at give op, må gerne hjælpe sit hold med at løse posten.\\
\newline
Oplever I deltagere, hvor I selv tænker, at det vil være bedst for deltageren at udgå at løbet, så kontakt udvalget. Det er kun deltagerne eller udvalget der kan træffe beslutning om at udgå af løbet.
\subsection{Følg med i løbet}
I app\'en kan I se, hvor langt holdene er nået og se om der er nogen på vej til jeres post.
\subsection{Postpokalen}
Alle løbets levende poster bliver vurderet af deltagerne. Posterne får point på en skala fra 1-10.\\
Den samlede score udregnes som et gennemsnit af de givne point. For at vi får den bedst mulige konkurrence, må I meget gerne minde deltagerne om at give jer point til postpokalen, inden de tjekker ud.\\
Undgå også at fedte for deltagerne med slik, snacks eller andet. - I giver heller ikke deltagerne fedtepoint for at fedte for jer.\\
\newline
Udover deltagerne, vil forskellige jury-medlemmer dukke op i løbet af løbet for at vurdere jeres post. Vi håber at I vil tage godt imod dem.
\subsection{Hemmeligheder}
En stor del af de point, deltagerne kan opnå på Sagaløbet, findes som hemmelige og skjulte point. Opdager I sådan nogle, så lad være med at fortælle deltagerne om dem - også de deltagere fra jeres egen kreds. På den måde får vi den mest fair konkurrence.
\newpage
\vspace*{.4cm}
\subsection{Mærker}
Når vi er færdige med løbet, kan alt postmandskab, som ikke allerede har et mærke, få et SagaCrew-mærke.\\
Alle deltagere som gennemfører Sagaløbet får også et mærke. Deltagere som tidligere har gennemført løbet, men kun fået ét mærke, kan købe ekstra mærker efter præmieoverrækkelsen til 10 kr. stykket.
\subsection{Kontakt}
Får I brug for at komme i kontakt med udvalget gennem natten, så ring til mål på tlf. 3267 9499. Har I brug for at få fat på et bestemt udvalgsmedlem, kan vores kontaktoplysninger findes på hjemmesiden. Vi ser dog helst at I kontakter mål frem for at kontakte enkelte udvalgsmedlemmer, da vi ordner koordinationen herfra.
\newline
Til sidst er der ikke så meget andet at sige end tak!\begin{itemize}
  \item fordi I er med til at skabe Sagaløbet 2025
  \item for at være gode ambassadører for vores arrangement
  \item for at tage jeres væbnere, seniorvæbnere og seniorer med på løb
  \item for at sikre at Sagaløbet er en fair konkurrence
  \item for at give deltagerne uforglemmelige oplevelser, som giver dem mulighed for at vokse
  \item for at hjælpe os med at udvikle løbet
  \item for at give os muligheden for at skabe fantastiske FDF-oplevelser i fællesskab
  \item for alt det vi ikke får sagt tak for
\end{itemize}
Vi glæder os helt vildt til at besøge jer i løbet af natten, og til at opleve løbet sammen med jer og deltagerne!
\newline
Rigtig godt løb!\\
\newline
\textcolor{søblå}{De bedste hilsner}\\
\textcolor{natblå}{\textbf{Alberte, Augusta, Charlotte, Dan, Jens Peter, Jonas, Lasse, Lucas, Magnus \& Thor}}\\
\textcolor{natblå}{\textbf{Sagaløbsudvalget}}\\
\newpage
\title{Postinfo}\\
Kære \textbf{Tårs}\\
\newline
Velkommen på Sagaløbet 2025.\\
Vi har glædet os vildt til at lave løb med jer.\\
\newline
I har fået \textbf{Post D3}. Posten er placeret på \textbf{RUTE D}\\
Vi har, som mange tidligere år, forsøgt at gøre løbet bedre for både deltagere og jer. I år drejer det sig om småjusteringer, så hvis I har været med før, så er det meste, som I kender det.\\
\newline
Hvis I ikke har været med før, så håber vi, at dette infobrev kan hjælpe jer med at være klædt godt på. Har I spørgsmål efter I har læst brevet, så sig til! Ring til mål på tlf. +45 3267 9499.
\subsection{Sagaløbets principper}
Til planlægningen af årets løb har vi skabt seks principper, som er det bærende element for både planlægning, afviklingen og udviklingen af Sagaløbet.\\
De seks principper er:
\begin{itemize}
  \item \textbf{Vi skaber Sagaløbet sammen}
  \item \textbf{Alle skal opleve at lykkes}
  \item \textbf{Alle skal opleve udfordringer}
  \item \textbf{Det kendte og ukendte}
  \item \textbf{Vi gør os umage}
  \item \textbf{Retfærdig konkurrence}
\end{itemize}
Vi håber, at I kan genkende at Sagaløbet er bygget op om disse seks bærende principper. Det bliver også med disse seks principper, at vi kommer til at træffe beslutninger under løbet. Har I spørgsmål til principperne, er vi altid klar til at tage en snak om dem.
\subsection{Ruter og deltagerstrøm}
Løbet er inddelt i fire mindre ruter. Hver rute er angivet med et bogstav og har 4 levende poster og én død post. Derudover er der 4 levende seniorposter i målområdet.\\
Ved at inddele løbet i fire mindre løb kan vi sprede deltagerne ud, have kontakt med dem løbende, give dem et helle og sikre et bedre flow i løbet.\\
\newline
Det betyder at I skal stå post fra 16:00 til 08:00.\\
Det er lang tid, og det kan være nødvendigt at hjælpe hinanden med at holde pauser undervejs.\\
\newline
Holdene er delt op i fire grupper med 12-13 hold i hver gruppe. Det betyder at I forhåbentlig når at se alle holdene fra en gruppe, før den næste dukker op.\\
For at holdet har ret til at løse jeres post, skal de være igang med jeres rute, og skal have løst de foregående ruter. I kan se om ruterne før jeres egen, har fået et hul på holdenes ID-kort.
\subsection{App og tjek ind}
På www.post25.sagalobet.dk skal I indberette holdenes point samt tjekke deltagerne ind og ud.\\
I skal logge ind med jeres postnummer - Altså \textbf{D3}\\
\newline
Kommer I til at lave en forkert registrering, skal I kontakte mål med det samme, på tlf. 3267 9499.
\newpage
\vspace*{.4cm}
\subsection{Postbeskrivelser}
I har fået udleveret 5 plader med postbeskrivelser til jeres post. På hver side af pladen er der en postbeskrivelse til enten væbner- eller seniorhold. I kan se på bjælken i toppen, om det er til væbner- eller seniorhold. Vi har i udgangspunktet ikke lavet andet end sproglige tilretninger på posten, men læs den lige igennem, så I er sikre på at vide det samme som deltagerne ved.\\
Vi vil \textbf{meget gerne} have pladerne retur i målområdet efter løbet.
\subsection{Sikkerhed på løbet}
For at give deltagerne en tryg oplevelse har vi nogle sikkerhedskrav til jeres post:
\begin{itemize}
  \item I skal opsætte jeres velkomstskilt, så det er det første deltagerne ser, når de kommer tæt på posten.
  \item I skal alle sammen bære de udleverede ID-kort yderst, så de er tydelige for deltagerne.
  \item I skal udlevere postbeskrivelserne til deltagerne, så de ved, at I er en ægte post.
  \item I skal huske at tjekke deltagerne ind og ud så snart de ankommer og forlader posten, så vi kan følge med i, hvor langt de er.
\end{itemize}
\subsection{Point}
Alle levende poster giver mellem 0 og 100 point, med mulighed for at optjene 25 bonuspoint, hvis posten løses på en bestemt måde. Giv kun deltagerne point ud fra deres opgaveløsning og ikke ud fra om de fedter for jer. Når et hold har gennemført posten, skal pointene indberettes i appen. Husk, at I ikke kan lave om i jeres registreringer, og det kun er muligt at indsende point én gang pr. hold.\\
\newline
Kommer I til at lave en forkert registrering, skal I kontakte mål med det samme, på tlf. 3267 9499.
\subsection{Åbningstider}
Alle poster åbner kl. 16:00 og lukker kl. 08:00.\\
Tjekker et hold ind på jeres post inden kl. 08.00, må de stadig gerne løse posten.\\
Hold, som ankommer efter kl. 08:00, kan ikke løse posten.\\
Basen lukker også kl. 08:00.
\subsection{Postforløb}
For at give deltagerne den samme oplevelse, vil vi gerne have jer til at følge dette postforløb:
\begin{itemize}
  \item Byd holdet velkommen til posten.
  \item Modtag holdets ID-kort og tjek dem ind på app\'en.
  \item Udlever postbeskrivelsen til holdet, hvis der ikke er kø, eller de er forrest i køen. Hvis holdet skal sidde i kø, skal I fortælle dem, hvor lang tid de kan forvente at vente.
  \item Sæt holdet i gang med at løse posten, når de har læst postbeskrivelsen.
  \item Fortæl holdet, hvor mange point de har tjent, når de er færdige med at løse posten.
  \item Registrer holdets point på app\'en.
  \item Tjek holdet ud på app\'en og giv dem deres ID-kort tilbage.
  \item Hold øje med om holdet er ved at gå kold. Giv dem noget opbakning, hvis det er tilfældet.
\end{itemize}
\newpage
\vspace*{.4cm}
\subsection{Baseområdet}
Hvis der er brug for at sende noget af jeres mandskab ind på en pause, kan de finde sig til rette i postområdet på basen. Husk at holdene kan komme hele natten, så sørg for ikke at sende hele postens mandskab afsted på en gang.\\
På basen er der en del af området, som er forbeholdt udvalget. Vi vil meget gerne hyggesnakke med jer og svare på spørgsmål, men vi har brug for ro i udvalgsområdet. Det håber, vi at I vil respektere. Har I brug for at nogen fra udvalget kommer ud, kan I sige det til de udvalgsmedlemmer, som bemander basebordet.
\subsection{Hvis en deltager vil udgå af løbet}
Undervejs på løbet kommer I til at opleve mange trætte børn og unge. Nogle af dem vil måske også gerne give op. Hvis det er tilfældet, skal I først spørge om de er sikre. Fortæl deltagerne, hvor langt der ca. er til næste post og hvem der bemander den. Det kan være, at det er deres egen kreds og at de gerne vil nå forbi den post også. Hvis deltageren stadig ønsker at give op, skal I finde et sted, de kan holde sig i læ og så ringe til mål på tlf. 3267 9499. Efterfølgende henter vi deltageren, når det er muligt. Deltagere som ønsker at give op, må gerne hjælpe sit hold med at løse posten.\\
\newline
Oplever I deltagere, hvor I selv tænker, at det vil være bedst for deltageren at udgå at løbet, så kontakt udvalget. Det er kun deltagerne eller udvalget der kan træffe beslutning om at udgå af løbet.
\subsection{Følg med i løbet}
I app\'en kan I se, hvor langt holdene er nået og se om der er nogen på vej til jeres post.
\subsection{Postpokalen}
Alle løbets levende poster bliver vurderet af deltagerne. Posterne får point på en skala fra 1-10.\\
Den samlede score udregnes som et gennemsnit af de givne point. For at vi får den bedst mulige konkurrence, må I meget gerne minde deltagerne om at give jer point til postpokalen, inden de tjekker ud.\\
Undgå også at fedte for deltagerne med slik, snacks eller andet. - I giver heller ikke deltagerne fedtepoint for at fedte for jer.\\
\newline
Udover deltagerne, vil forskellige jury-medlemmer dukke op i løbet af løbet for at vurdere jeres post. Vi håber at I vil tage godt imod dem.
\subsection{Hemmeligheder}
En stor del af de point, deltagerne kan opnå på Sagaløbet, findes som hemmelige og skjulte point. Opdager I sådan nogle, så lad være med at fortælle deltagerne om dem - også de deltagere fra jeres egen kreds. På den måde får vi den mest fair konkurrence.
\newpage
\vspace*{.4cm}
\subsection{Mærker}
Når vi er færdige med løbet, kan alt postmandskab, som ikke allerede har et mærke, få et SagaCrew-mærke.\\
Alle deltagere som gennemfører Sagaløbet får også et mærke. Deltagere som tidligere har gennemført løbet, men kun fået ét mærke, kan købe ekstra mærker efter præmieoverrækkelsen til 10 kr. stykket.
\subsection{Kontakt}
Får I brug for at komme i kontakt med udvalget gennem natten, så ring til mål på tlf. 3267 9499. Har I brug for at få fat på et bestemt udvalgsmedlem, kan vores kontaktoplysninger findes på hjemmesiden. Vi ser dog helst at I kontakter mål frem for at kontakte enkelte udvalgsmedlemmer, da vi ordner koordinationen herfra.
\newline
Til sidst er der ikke så meget andet at sige end tak!\begin{itemize}
  \item fordi I er med til at skabe Sagaløbet 2025
  \item for at være gode ambassadører for vores arrangement
  \item for at tage jeres væbnere, seniorvæbnere og seniorer med på løb
  \item for at sikre at Sagaløbet er en fair konkurrence
  \item for at give deltagerne uforglemmelige oplevelser, som giver dem mulighed for at vokse
  \item for at hjælpe os med at udvikle løbet
  \item for at give os muligheden for at skabe fantastiske FDF-oplevelser i fællesskab
  \item for alt det vi ikke får sagt tak for
\end{itemize}
Vi glæder os helt vildt til at besøge jer i løbet af natten, og til at opleve løbet sammen med jer og deltagerne!
\newline
Rigtig godt løb!\\
\newline
\textcolor{søblå}{De bedste hilsner}\\
\textcolor{natblå}{\textbf{Alberte, Augusta, Charlotte, Dan, Jens Peter, Jonas, Lasse, Lucas, Magnus \& Thor}}\\
\textcolor{natblå}{\textbf{Sagaløbsudvalget}}\\
\newpage
\title{Postinfo}\\
Kære \textbf{Øster Hassing}\\
\newline
Velkommen på Sagaløbet 2025.\\
Vi har glædet os vildt til at lave løb med jer.\\
\newline
I har fået \textbf{Post D4}. Posten er placeret på \textbf{RUTE D}\\
Vi har, som mange tidligere år, forsøgt at gøre løbet bedre for både deltagere og jer. I år drejer det sig om småjusteringer, så hvis I har været med før, så er det meste, som I kender det.\\
\newline
Hvis I ikke har været med før, så håber vi, at dette infobrev kan hjælpe jer med at være klædt godt på. Har I spørgsmål efter I har læst brevet, så sig til! Ring til mål på tlf. +45 3267 9499.
\subsection{Sagaløbets principper}
Til planlægningen af årets løb har vi skabt seks principper, som er det bærende element for både planlægning, afviklingen og udviklingen af Sagaløbet.\\
De seks principper er:
\begin{itemize}
  \item \textbf{Vi skaber Sagaløbet sammen}
  \item \textbf{Alle skal opleve at lykkes}
  \item \textbf{Alle skal opleve udfordringer}
  \item \textbf{Det kendte og ukendte}
  \item \textbf{Vi gør os umage}
  \item \textbf{Retfærdig konkurrence}
\end{itemize}
Vi håber, at I kan genkende at Sagaløbet er bygget op om disse seks bærende principper. Det bliver også med disse seks principper, at vi kommer til at træffe beslutninger under løbet. Har I spørgsmål til principperne, er vi altid klar til at tage en snak om dem.
\subsection{Ruter og deltagerstrøm}
Løbet er inddelt i fire mindre ruter. Hver rute er angivet med et bogstav og har 4 levende poster og én død post. Derudover er der 4 levende seniorposter i målområdet.\\
Ved at inddele løbet i fire mindre løb kan vi sprede deltagerne ud, have kontakt med dem løbende, give dem et helle og sikre et bedre flow i løbet.\\
\newline
Det betyder at I skal stå post fra 16:00 til 08:00.\\
Det er lang tid, og det kan være nødvendigt at hjælpe hinanden med at holde pauser undervejs.\\
\newline
Holdene er delt op i fire grupper med 12-13 hold i hver gruppe. Det betyder at I forhåbentlig når at se alle holdene fra en gruppe, før den næste dukker op.\\
For at holdet har ret til at løse jeres post, skal de være igang med jeres rute, og skal have løst de foregående ruter. I kan se om ruterne før jeres egen, har fået et hul på holdenes ID-kort.
\subsection{App og tjek ind}
På www.post25.sagalobet.dk skal I indberette holdenes point samt tjekke deltagerne ind og ud.\\
I skal logge ind med jeres postnummer - Altså \textbf{D4}\\
\newline
Kommer I til at lave en forkert registrering, skal I kontakte mål med det samme, på tlf. 3267 9499.
\newpage
\vspace*{.4cm}
\subsection{Postbeskrivelser}
I har fået udleveret 5 plader med postbeskrivelser til jeres post. På hver side af pladen er der en postbeskrivelse til enten væbner- eller seniorhold. I kan se på bjælken i toppen, om det er til væbner- eller seniorhold. Vi har i udgangspunktet ikke lavet andet end sproglige tilretninger på posten, men læs den lige igennem, så I er sikre på at vide det samme som deltagerne ved.\\
Vi vil \textbf{meget gerne} have pladerne retur i målområdet efter løbet.
\subsection{Sikkerhed på løbet}
For at give deltagerne en tryg oplevelse har vi nogle sikkerhedskrav til jeres post:
\begin{itemize}
  \item I skal opsætte jeres velkomstskilt, så det er det første deltagerne ser, når de kommer tæt på posten.
  \item I skal alle sammen bære de udleverede ID-kort yderst, så de er tydelige for deltagerne.
  \item I skal udlevere postbeskrivelserne til deltagerne, så de ved, at I er en ægte post.
  \item I skal huske at tjekke deltagerne ind og ud så snart de ankommer og forlader posten, så vi kan følge med i, hvor langt de er.
\end{itemize}
\subsection{Point}
Alle levende poster giver mellem 0 og 100 point, med mulighed for at optjene 25 bonuspoint, hvis posten løses på en bestemt måde. Giv kun deltagerne point ud fra deres opgaveløsning og ikke ud fra om de fedter for jer. Når et hold har gennemført posten, skal pointene indberettes i appen. Husk, at I ikke kan lave om i jeres registreringer, og det kun er muligt at indsende point én gang pr. hold.\\
\newline
Kommer I til at lave en forkert registrering, skal I kontakte mål med det samme, på tlf. 3267 9499.
\subsection{Åbningstider}
Alle poster åbner kl. 16:00 og lukker kl. 08:00.\\
Tjekker et hold ind på jeres post inden kl. 08.00, må de stadig gerne løse posten.\\
Hold, som ankommer efter kl. 08:00, kan ikke løse posten.\\
Basen lukker også kl. 08:00.
\subsection{Postforløb}
For at give deltagerne den samme oplevelse, vil vi gerne have jer til at følge dette postforløb:
\begin{itemize}
  \item Byd holdet velkommen til posten.
  \item Modtag holdets ID-kort og tjek dem ind på app\'en.
  \item Udlever postbeskrivelsen til holdet, hvis der ikke er kø, eller de er forrest i køen. Hvis holdet skal sidde i kø, skal I fortælle dem, hvor lang tid de kan forvente at vente.
  \item Sæt holdet i gang med at løse posten, når de har læst postbeskrivelsen.
  \item Fortæl holdet, hvor mange point de har tjent, når de er færdige med at løse posten.
  \item Registrer holdets point på app\'en.
  \item Tjek holdet ud på app\'en og giv dem deres ID-kort tilbage.
  \item Hold øje med om holdet er ved at gå kold. Giv dem noget opbakning, hvis det er tilfældet.
\end{itemize}
\newpage
\vspace*{.4cm}
\subsection{Baseområdet}
Hvis der er brug for at sende noget af jeres mandskab ind på en pause, kan de finde sig til rette i postområdet på basen. Husk at holdene kan komme hele natten, så sørg for ikke at sende hele postens mandskab afsted på en gang.\\
På basen er der en del af området, som er forbeholdt udvalget. Vi vil meget gerne hyggesnakke med jer og svare på spørgsmål, men vi har brug for ro i udvalgsområdet. Det håber, vi at I vil respektere. Har I brug for at nogen fra udvalget kommer ud, kan I sige det til de udvalgsmedlemmer, som bemander basebordet.
\subsection{Hvis en deltager vil udgå af løbet}
Undervejs på løbet kommer I til at opleve mange trætte børn og unge. Nogle af dem vil måske også gerne give op. Hvis det er tilfældet, skal I først spørge om de er sikre. Fortæl deltagerne, hvor langt der ca. er til næste post og hvem der bemander den. Det kan være, at det er deres egen kreds og at de gerne vil nå forbi den post også. Hvis deltageren stadig ønsker at give op, skal I finde et sted, de kan holde sig i læ og så ringe til mål på tlf. 3267 9499. Efterfølgende henter vi deltageren, når det er muligt. Deltagere som ønsker at give op, må gerne hjælpe sit hold med at løse posten.\\
\newline
Oplever I deltagere, hvor I selv tænker, at det vil være bedst for deltageren at udgå at løbet, så kontakt udvalget. Det er kun deltagerne eller udvalget der kan træffe beslutning om at udgå af løbet.
\subsection{Følg med i løbet}
I app\'en kan I se, hvor langt holdene er nået og se om der er nogen på vej til jeres post.
\subsection{Postpokalen}
Alle løbets levende poster bliver vurderet af deltagerne. Posterne får point på en skala fra 1-10.\\
Den samlede score udregnes som et gennemsnit af de givne point. For at vi får den bedst mulige konkurrence, må I meget gerne minde deltagerne om at give jer point til postpokalen, inden de tjekker ud.\\
Undgå også at fedte for deltagerne med slik, snacks eller andet. - I giver heller ikke deltagerne fedtepoint for at fedte for jer.\\
\newline
Udover deltagerne, vil forskellige jury-medlemmer dukke op i løbet af løbet for at vurdere jeres post. Vi håber at I vil tage godt imod dem.
\subsection{Hemmeligheder}
En stor del af de point, deltagerne kan opnå på Sagaløbet, findes som hemmelige og skjulte point. Opdager I sådan nogle, så lad være med at fortælle deltagerne om dem - også de deltagere fra jeres egen kreds. På den måde får vi den mest fair konkurrence.
\newpage
\vspace*{.4cm}
\subsection{Mærker}
Når vi er færdige med løbet, kan alt postmandskab, som ikke allerede har et mærke, få et SagaCrew-mærke.\\
Alle deltagere som gennemfører Sagaløbet får også et mærke. Deltagere som tidligere har gennemført løbet, men kun fået ét mærke, kan købe ekstra mærker efter præmieoverrækkelsen til 10 kr. stykket.
\subsection{Kontakt}
Får I brug for at komme i kontakt med udvalget gennem natten, så ring til mål på tlf. 3267 9499. Har I brug for at få fat på et bestemt udvalgsmedlem, kan vores kontaktoplysninger findes på hjemmesiden. Vi ser dog helst at I kontakter mål frem for at kontakte enkelte udvalgsmedlemmer, da vi ordner koordinationen herfra.
\newline
Til sidst er der ikke så meget andet at sige end tak!\begin{itemize}
  \item fordi I er med til at skabe Sagaløbet 2025
  \item for at være gode ambassadører for vores arrangement
  \item for at tage jeres væbnere, seniorvæbnere og seniorer med på løb
  \item for at sikre at Sagaløbet er en fair konkurrence
  \item for at give deltagerne uforglemmelige oplevelser, som giver dem mulighed for at vokse
  \item for at hjælpe os med at udvikle løbet
  \item for at give os muligheden for at skabe fantastiske FDF-oplevelser i fællesskab
  \item for alt det vi ikke får sagt tak for
\end{itemize}
Vi glæder os helt vildt til at besøge jer i løbet af natten, og til at opleve løbet sammen med jer og deltagerne!
\newline
Rigtig godt løb!\\
\newline
\textcolor{søblå}{De bedste hilsner}\\
\textcolor{natblå}{\textbf{Alberte, Augusta, Charlotte, Dan, Jens Peter, Jonas, Lasse, Lucas, Magnus \& Thor}}\\
\textcolor{natblå}{\textbf{Sagaløbsudvalget}}\\
\newpage

